\documentclass[DIN, pagenumber=false, fontsize=11pt, parskip=half]{scrartcl}

\usepackage{amsmath}
\usepackage{amsfonts}
\usepackage{amssymb}
\usepackage{enumitem}
\usepackage[utf8]{inputenc} % this is needed for umlauts
\usepackage[ngerman]{babel} % this is needed for umlauts
\usepackage[T1]{fontenc} 
\usepackage{commath}
\usepackage{xcolor}
\usepackage{booktabs}
\usepackage{float}
\usepackage{tikz-timing}
\usepackage{tikz}
\usepackage{multirow}

\usetikzlibrary{calc,shapes.multipart,chains,arrows}

\title{Grundlagen der Rechnerarchitektur}
\author{Tim Luchterhand, Paul Nykiel (Abgabegruppe 117)}

\newcommand{\boolshitKgV}[1]{\text{kgV}(#1,30) &=& 30\\}
\newcommand{\boolshitggT}[1]{\text{ggT}(#1,30) &=& #1\\}

\begin{document}
    \maketitle
    \textbf{Hinweis: } Jede Zahl, bei der keine Basis spezifiziert ist, ist im 10er System zu interpretieren, sofern nicht anders angegeben.
    \section{Boolesche Algebra}
    \subsection{}
    Beweis durch Nachrechnen:
    \subsubsection{Absorbierendes Element}
    \begin{eqnarray*}
        \boolshitKgV{1}
        \boolshitKgV{2}
        \boolshitKgV{3}
        \boolshitKgV{5}
        \boolshitKgV{6}
        \boolshitKgV{10}
        \boolshitKgV{15}
    \end{eqnarray*}
    \subsubsection{Neutrales Element}
    \begin{eqnarray*}
        \boolshitggT{1}
        \boolshitggT{2}
        \boolshitggT{3}
        \boolshitggT{5}
        \boolshitggT{6}
        \boolshitggT{10}
        \boolshitggT{15}
    \end{eqnarray*}

    \subsection{}
    \begin{enumerate}[label=(\alph*)]
        \item 
            \begin{eqnarray*}
                \overline{x_1} \cdot \overline{x_2} &=& \overline{x_1 + x_2}\\
                \Leftrightarrow (\overline{x_1} \cdot \overline{x_2}) \cdot (x_1 + x_2) &=& \overline{(x_1 + x_2)} \cdot  (x_1 + x_2)\\
                \stackrel{\text{H4'}}{\Leftrightarrow} (\overline{x_1} \cdot \overline{x_2}) \cdot x_1 +  (\overline{x_1} \cdot \overline{x_2}) \cdot x_2 &=& 0\\
                \Leftrightarrow (\overline{x_1} \cdot x_1) \cdot \overline{x_2} + (\overline{x_2} \cdot x_2)  \cdot \overline{x_1} &=& 0\\
                \stackrel{\text{H4'}}{\Leftrightarrow} 0 \cdot \overline{x_2} + 0 \cdot \overline{x_1} &=& 0\\
                \Leftrightarrow 0 + 0 &=& 0\\
                \stackrel{\text{H3}}{\Leftrightarrow} 0 &=& 0
            \end{eqnarray*}
        \item
            Es wird $\overline{\overline{a}} = a$ genutzt, Beweis siehe Vorlesung.
            \begin{equation*}
                \overline{x_1} + \overline{x_2} = \overline{\overline{\overline{x_1} + \overline{x_2}}} \stackrel{\text{(a)}}{=}
                \overline{\overline{\overline{x_1}} \cdot \overline{\overline{x_2}}} = \overline{x_1 \cdot x_2}
            \end{equation*}
    \end{enumerate}

    \section{Minimierung Boolescher Funktionen}
    \setcounter{subsection}{2}
    \subsection{}
    \begin{enumerate}[label=(\alph*)]
        \item
            \begin{equation*}
                f(x_1) = \overline{x_1} \cdot (x_1 + (x_1 \cdot \overline{x_1})) \stackrel{\text{P9}}{=} \overline{x_1} \cdot (x_1 + 0) \stackrel{\text{P5'}}{=} \overline{x_1} \cdot x_1 \stackrel{\text{P9}}{=} 0
            \end{equation*}
        \item 
            \begin{equation*}
                f(x_1, x_2) = x_1 \cdot x_2 + x_1 \cdot (\overline{x_2} \cdot 1) \stackrel{\text{P5}}{=} x_1 \cdot x_2 + x_1 \cdot \overline{x_2} \stackrel{\text{P4}}{=} x_1 \cdot (x_2 + \overline{x_2})
                \stackrel{\text{P9'}}{=} x_1 \cdot 1 \stackrel{\text{P5}}{=} x_1
            \end{equation*}
        \item 
            \begin{equation*}
                f(x_1, x_2) = x_1 \cdot (\overline{x_1} \cdot (\overline{\overline{x_2}} + \overline{x_2})) \stackrel{\text{P7}}{=} 
                    x_1 \cdot (\overline{x_1} \cdot (x_2 + \overline{x_2})) \stackrel{\text{P9'}}{=} x_1 \cdot (\overline{x_1} \cdot 1) \stackrel{\text{P5}}{=} 
                    x_1 \cdot \overline{x_1} \stackrel{\text{P9}}{=} 0
            \end{equation*}
        \item 
            \begin{equation*}
                f(x_1, x_2, x_3) = \overline{(x_1 \cdot x_2)} + ((x_3 + 0) \cdot \overline{x_3})
                \stackrel{\text{P5'}}{=} \overline{(x_1 \cdot x_2)} + (x_3 \cdot \overline{x_3})
                \stackrel{\text{P9}}{=} \overline{(x_1 \cdot x_2)} + 0
                \stackrel{\text{P5*}}{=} \overline{(x_1 \cdot x_2)}
            \end{equation*}
        \item 
            \begin{eqnarray*}
                f(x_1, x_2, x_3, x_4) &=& (\overline{x_1} \cdot (x_3 + \overline{x_3}) \cdot x_2) + (\overline{x_1} \cdot x_3) + \overline{(x_1 + x_4)} \\
                &\stackrel{\text{P9'}}{=}& (\overline{x_1} \cdot 1 \cdot x_2) + (\overline{x_1} \cdot x_3) + \overline{(x_1 + x_4)} \\
                &\stackrel{\text{P5}}{=}& \overline{x_1} \cdot x_2 + \overline{x_1} \cdot x_3 + \overline{x_1} \cdot \overline{x_4} \\
                &\stackrel{\text{P4}}{=}& \overline{x_1} \cdot (x_2 + x_3 +  \overline{x_4})
            \end{eqnarray*}
    \end{enumerate}

    \subsection{}
    \begin{enumerate}
        \item
            \begin{eqnarray*}
                &&\overline{(\overline{x_1} + \overline{x_2}) \cdot (x_1 + \overline{x_3}) \cdot (x_2 + x_3)} \\
                &\stackrel{\text{P8}}{=}&  \overline{(\overline{x_1} + \overline{x_2}) \cdot (x_1 + \overline{x_3})} + \overline{(x_2 + x_3)}  \\
                &\stackrel{\text{P8}}{=}&  \overline{(\overline{x_1} + \overline{x_2})} + \overline{(x_1 + \overline{x_3})} + \overline{(x_2 + x_3)}  \\
                &\stackrel{\text{P8'}}{=}&  (\overline{\overline{x_1}} \cdot \overline{\overline{x_2}}) \cdot 
                                        (\overline{x_1} \cdot  \overline{\overline{x_3}}) \cdot (\overline{x_2} \cdot \overline{x_3}) \\
                &\stackrel{\text{P7}}{=}& (x_1 \cdot x_2) \cdot (\overline{x_1} \cdot x_3) \cdot (\overline{x_2} \cdot \overline{x_3})
            \end{eqnarray*}
        \item
            \begin{eqnarray*}
                &&\overline{\overline{(x_1 \cdot \overline{(x_2 \cdot x_2)})} \cdot \overline{(\overline{(x_1 \cdot x_1)}\cdot x_2 )}} \\
                &\stackrel{\text{P8}}{=}& \overline{\overline{(x_1 \cdot \overline{(x_2 \cdot x_2)})}} + \overline{\overline{(\overline{(x_1 \cdot x_1)} \cdot x_2)}} \\
                &\stackrel{\text{P7}}{=}& (x_1 \cdot \overline{(x_2 \cdot x_2)}) + (\overline{(x_1 \cdot x_1)} \cdot x_2)\\
                &\stackrel{\text{P3}}{=}& x_1 \cdot \overline{x_2} + \overline{x_1} \cdot x_2\\
            \end{eqnarray*}
    \end{enumerate}

    \section{Shannon Zerlegung und Erweiterung}
    \setcounter{subsection}{4}
    \subsection{}
    \begin{enumerate}[label=(\alph*)]
        \item
            \begin{eqnarray*}
                f(x_1, x_2) &=& x_1 \cdot (1 \cdot x_2 + 1 \cdot \overline{x_2} + 0 \cdot x_2)
                + \overline{x_1} \cdot (0 \cdot x_2 + 0 \cdot \overline{x_2} + 1 \cdot x_2) \\
                &=& x_1 \cdot (x_2 + \overline{x_2}) + \overline{x_1} \cdot (x_2) \\
                &=& x_1 \cdot (x_2 \cdot (1 + 0) + \overline{x_2} \cdot (0 + 1)) + \overline{x_1} \cdot (x_2 \cdot (1) + \overline{x_2} \cdot (0)) \\
                &=& x_1 + \overline{x_1} \cdot x_2 
            \end{eqnarray*}
        \item
            \begin{eqnarray*}
                f(x_1, x_2, x_3) &=& x_1 \cdot (x_2 x_3 + x_2 \overline{x_3}) + \overline{x_1} \cdot (\overline{x_2} x_3) \\
                &=& x_1 \cdot (x_2 \cdot (x_3 + \overline{x_3})) + \overline{x_1} \cdot (\overline{x_2} x_3) = x_1 \cdot x_2 + \overline{x_1} \cdot (\overline{x_2} x_3) \\
                &=& x_1 x_2 + \overline{x_1} \ \overline{x_2} x_3
            \end{eqnarray*}
        \item
            \begin{eqnarray*}
                f(x_1, x_2, x_3) &=& x_1 \cdot (x_2 x_3 + \overline{x_2} x_3) + \overline{x_1} \cdot (x_2 \overline{x_3} + \overline{x_2} x_3 + \overline{x_2} \ \overline{x_3}) \\
                &=& x_1 \cdot (x_2 x_3 + \overline{x_2} x_3) + \overline{x_1} \cdot (x_2 \cdot (\overline{x_3}) + \overline{x_2} \cdot (x_3 + \overline{x_3})) \\
                &=& x_1 \cdot (x_3 \cdot(x_2 + \overline{x_2})) + \overline{x_1} \cdot (x_2 \cdot (\overline{x_3}) + \overline{x_2} \cdot (x_3 + \overline{x_3})) \\
                &=& x_1 x_3 + \overline{x_1} \cdot (x_2 \overline{x_3} + \overline{x_2}) = x_1 x_3 + \overline{x_1} x_2 \overline{x_3} + \overline{x_1} \ \overline{x_2}
            \end{eqnarray*}
    \end{enumerate}

    \subsection{}
    \begin{enumerate}[label=(\alph*)]
        \item 
            \begin{eqnarray*}
                f(x_1, x_2, x_3) &=& f(0, 0, 0) \overline{x_1} \ \overline{x_2} \ \overline{x_3} + f(0, 0, 1) \overline{x_1} \ \overline{x_2} x_3 \\ 
                &+& f(0, 1, 0) \overline{x_1} x_2 \ \overline{x_3} + f(0, 1, 1) \overline{x_1} x_2 x_3\\
                &+& f(1, 0, 0) x_1 \overline{x_2} \  \overline{x_3} + f(1, 0, 1) x_1 \overline{x_2} x_3\\
                &+& f(1, 1, 0) x_1 x_2 \overline{x_3} + f(1, 1, 1) x_1 x_2 x_3\\
                &=& \overline{x_1} \ \overline{x_2} \ \overline{x_3} + \overline{x_1} x_2 \ \overline{x_3} + x_1 \overline{x_2} \  \overline{x_3} + x_1 x_2 \overline{x_3} + x_1 x_2 x_3
            \end{eqnarray*}
        \item 
            \begin{eqnarray*}
                f(x_1, x_2, x_3, x_4) &=& f(0, 0, 0, 0) \overline{x_1} \ \overline{x_2} \ \overline{x_3} \ \overline{x_4} \\
                &&+ f(0, 0, 0, 1) \overline{x_1} \ \overline{x_2} \ \overline{x_3} x_4\\
                &&+ f(0, 0, 1, 0) \overline{x_1} \ \overline{x_2} x_3 \overline{x_4}\\
                &&+ f(0, 0, 1, 1) \overline{x_1} \ \overline{x_2} x_3 x_4\\
                &&+ f(0, 1, 0, 0) \overline{x_1} x_2 \overline{x_3} \ \overline{x_4}\\
                &&+ f(0, 1, 0, 1) \overline{x_1} x_2 \overline{x_3} x_4\\
                &&+ f(0, 1, 1, 0) \overline{x_1} x_2 x_3 \overline{x_4}\\
                &&+ f(0, 1, 1, 1) \overline{x_1} x_2 x_3 x_4\\
                &&+ f(1, 0, 0, 0) x_1 \overline{x_2} \ \overline{x_3} \ \overline{x_4}\\
                &&+ f(1, 0, 0, 1) x_1 \overline{x_2} \ \overline{x_3} x_4\\
                &&+ f(1, 0, 1, 0) x_1 \overline{x_2} x_3 \overline{x_4}\\
                &&+ f(1, 0, 1, 1) x_1 \overline{x_2} x_3 x_4\\
                &&+ f(1, 1, 0, 0) x_1 x_2 \overline{x_3} \ \overline{x_4}\\
                &&+ f(1, 1, 0, 1) x_1 x_2 \overline{x_3} x_4\\
                &&+ f(1, 1, 1, 0) x_1 x_2 x_3 \overline{x_4}\\
                &&+ f(1, 1, 1, 1) x_1 x_2 x_3 x_4\\
                &=&  \overline{x_1} \ \overline{x_2} \ \overline{x_3} x_4\\
                &&+ \overline{x_1} \ \overline{x_2} x_3 x_4\\
                &&+ \overline{x_1} x_2 \overline{x_3} x_4\\
                &&+ \overline{x_1} x_2 x_3 x_4\\
                &&+ x_1 x_2 x_3 \overline{x_4}\\
                &&+ x_1 x_2 x_3 x_4\\
            \end{eqnarray*}
    \end{enumerate}
\end{document}
