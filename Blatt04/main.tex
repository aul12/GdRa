\documentclass[DIN, pagenumber=false, fontsize=11pt, parskip=half]{scrartcl}

\usepackage{amsmath}
\usepackage{amsfonts}
\usepackage{amssymb}
\usepackage{enumitem}
\usepackage[utf8]{inputenc} % this is needed for umlauts
\usepackage[ngerman]{babel} % this is needed for umlauts
\usepackage[T1]{fontenc} 
\usepackage{commath}
\usepackage{xcolor}
\usepackage{booktabs}
\usepackage{float}
\usepackage{tikz-timing}
\usepackage{tikz}
\usepackage{multirow}

\usetikzlibrary{calc,shapes.multipart,chains,arrows}

\title{Grundlagen der Rechnerarchitektur}
\author{Tim Luchterhand, Paul Nykiel (Abgabegruppe 117)}

\newcommand{\boolshitKgV}[1]{\text{kgV}(#1,30) &=& 30\\}
\newcommand{\boolshitggT}[1]{\text{ggT}(#1,30) &=& #1\\}

\begin{document}
    \maketitle
    \textbf{Hinweis: } Jede Zahl, bei der keine Basis spezifiziert ist, ist im 10er System zu interpretieren, sofern nicht anders angegeben.
    \section{Boolesche Algebra}
    \subsection{}
    \subsubsection{Absorbierendes Element}
    \begin{eqnarray*}
        \boolshitKgV{1}
        \boolshitKgV{2}
        \boolshitKgV{3}
        \boolshitKgV{5}
        \boolshitKgV{6}
        \boolshitKgV{10}
        \boolshitKgV{15}
    \end{eqnarray*}
    \subsubsection{Neutrales Element}
    \begin{eqnarray*}
        \boolshitggT{1}
        \boolshitggT{2}
        \boolshitggT{3}
        \boolshitggT{5}
        \boolshitggT{6}
        \boolshitggT{10}
        \boolshitggT{15}
    \end{eqnarray*}

    \subsection{}
    \begin{enumerate}[label=(\alph*)]
        \item 
            \begin{eqnarray*}
                \overline{x_1} \cdot \overline{x_2} &=& \overline{x_1 + x_2}\\
                \Leftrightarrow (\overline{x_1} \cdot \overline{x_2}) \cdot (x_1 + x_2) &=& \overline{(x_1 + x_2)} \cdot  (x_1 + x_2)\\
                \Leftrightarrow (\overline{x_1} \cdot \overline{x_2}) \cdot x_1 +  (\overline{x_1} \cdot \overline{x_2}) \cdot x_2 &=& 0\\
                \Leftrightarrow (\overline{x_1} \cdot x_1) \cdot \overline{x_2} + (\overline{x_2} \cdot x_2)  \cdot \overline{x_1} &=& 0\\
                \Leftrightarrow 0 \cdot \overline{x_1} + 0 \cdot \overline{x_1} &=& 0\\
                \Leftrightarrow 0 + 0 &=& 0\\
                \Leftrightarrow 0 &=& 0
            \end{eqnarray*}
        \item
            \begin{equation*}
                \overline{x_1} + \overline{x_2} = \overline{\overline{\overline{x_1} + \overline{x_2}}} \stackrel{\text{(a)}}{=}
                \overline{\overline{\overline{x_1}} \cdot \overline{\overline{x_2}}} = \overline{x_1 \cdot x_2}
            \end{equation*}
    \end{enumerate}

    \section{Minimierung Boolescher Funktionen}
    \setcounter{subsection}{2}
    \subsection{}
    \begin{enumerate}[label=(\alph*)]
        \item
            \begin{equation*}
                f(x_1) = \overline{x_1} \cdot (x_1 + (x_1 \cdot \overline{x_1})) = \overline{x_1} \cdot (x_1 + 0) = \overline{x_1} \cdot x_1 = 0
            \end{equation*}
        \item 
            \begin{equation*}
                f(x_1, x_2) = x_1 \cdot x_2 + x_1 \cdot (\overline{x_2} \cdot 1) = x_1 \cdot x_2 + x_1 \cdot \overline{x_2} = x_1 \cdot (x_2 + \overline{x_2})
                = x_1 \cdot 1 = x_1
            \end{equation*}
        \item 
            \begin{equation*}
                f(x_1, x_2) = x_1 \cdot (\overline{x_1} \cdot (\overline{\overline{x_2}} + \overline{x_2})) = 
                    x_1 \cdot (\overline{x_1} \cdot (x_2 + \overline{x_2})) = x_1 \cdot (\overline{x_1} \cdot 1) =
                    x_1 \cdot \overline{x_1} = 0
            \end{equation*}
        \item 
            \begin{equation*}
                f(x_1, x_2, x_3) = \overline{(x_1 \cdot x_2)} + ((x_3 + 0) \cdot \overline{x_3})
                = \overline{(x_1 \cdot x_2)} + (x_3 \cdot \overline{x_3})
                = \overline{(x_1 \cdot x_2)} + 0
                = \overline{(x_1 \cdot x_2)}
            \end{equation*}
        \item 
            \begin{eqnarray*}
                f(x_1, x_2, x_3, x_4) &=& (\overline{x_1} \cdot (x_3 + \overline{x_3}) \cdot x_2) + (\overline{x_1} \cdot x_3) + \overline{(x_1 + x_4)} \\
                &=& (\overline{x_1} \cdot 1 \cdot x_2) + (\overline{x_1} \cdot x_3) + \overline{(x_1 + x_4)} \\
                &=& \overline{x_1} \cdot x_2 + \overline{x_1} \cdot x_3 + \overline{x_1} \cdot \overline{x_4} \\
                &=& \overline{x_1} \cdot (x_2 + x_3 +  \overline{x_4})
            \end{eqnarray*}
    \end{enumerate}

    \section{Shannon Zerlegung und Erweiterung}
    \setcounter{subsection}{3}
    \begin{enumerate}[label=(\alph*)]
        \item 
            \begin{eqnarray*}
                f(x_1, x_2) &=& x_1 \cdot f(1, x_2) + x_2 \cdot f(x_1, 1) \\
                &=& x_1 \cdot (1 \cdot x_2 + 1 \cdot \overline{x_2} + 0 \cdot x_2) + x_2 \cdot (x_1 \cdot 1 + x_1 \cdot 0 + \overline{x_1} \cdot 1) \\
                &=& x_1 \cdot (x_2 + \overline{x_2} + 0) + x_2 \cdot (x_1 + \overline{x_1}) \\
                &=& x_1 \cdot 1 + x_2 \cdot 1 \\
                &=& x_1 + x_2 
            \end{eqnarray*}
        \item 
            \begin{eqnarray*}
                f(x_1, x_2, x_3) &=& x_1 \cdot f(1, x_2, x_3) + x_2 \cdot f(x_1, 1, x_2) + x_3 \cdot f(x_1, x_2, 1) \\
                x_1 \cdot (
            \end{eqnarray*}
    \end{enumerate}

\end{document}
