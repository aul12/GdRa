\documentclass[DIN, pagenumber=false, fontsize=11pt, parskip=half]{scrartcl}

\usepackage{amsmath}
\usepackage{amsfonts}
\usepackage{amssymb}
\usepackage{enumitem}
\usepackage[utf8]{inputenc} % this is needed for umlauts
\usepackage[ngerman]{babel} % this is needed for umlauts
\usepackage[T1]{fontenc} 
\usepackage{commath}
\usepackage{xcolor}
\usepackage{booktabs}
\usepackage{float}
\usepackage{tikz-timing}
\usepackage{tikz}
\usepackage{multirow}
\usepackage{colortbl}
\usepackage{xstring}
\usepackage{circuitikz}

\usetikzlibrary{calc,shapes.multipart,chains,arrows}

\title{Grundlagen der Rechnerarchitektur}
\author{Tim Luchterhand, Paul Nykiel (Abgabegruppe 117)}

\newcommand{\boolshitKgV}[1]{\text{kgV}(#1,30) &=& 30\\}
\newcommand{\boolshitggT}[1]{\text{ggT}(#1,30) &=& #1\\}
\newcommand{\minTerm}[4]{
    \IfEqCase{#1}{
        {0}{\overline{x_0}\ }
        {1}{x_0}
    }
    \IfEqCase{#2}{
        {0}{\overline{x_1}\ }
        {1}{x_1}
    }
    \IfEqCase{#3}{
        {0}{\overline{x_2}\ }
        {1}{x_2}
    }
    \IfEqCase{#4}{
        {0}{\overline{x_3}\ }
        {1}{x_3}
    }[\PackageError{minTerm}{Values must either be 0 or 1}]
}

\newcommand{\maxTerm}[4]{
    (
    \IfEqCase{#1}{
        {0}{\overline{x_0} + }
        {1}{x_0 + }
    }
    \IfEqCase{#2}{
        {0}{\overline{x_1} + }
        {1}{x_1 + }
    }
    \IfEqCase{#3}{
        {0}{\overline{x_2} + }
        {1}{x_2 + }
    }
    \IfEqCase{#4}{
        {0}{\overline{x_3}}
        {1}{x_3}
    }[\PackageError{minTerm}{Values must either be 0 or 1}]
    )
}

\begin{document}
    \maketitle
    \section{Umcodierer}
    \subsection{}
    \begin{enumerate}[label=(\alph*)]
        \item
			Wahrheitstabelle:
            \begin{figure}[H]
                \centering
                \begin{tabular}{ccc|cccccccc}
                    \toprule
                    $x_2$ &$x_1$ & $x_0$ & $y_7$ & $y_6$ & $y_5$ & $y_4$ & $y_3$ & $y_2$ & $y_1$ & $y_0$ \\
                    \midrule
                    0 & 0 & 0  &  0 & 0 & 0 & 0 & 0 & 0 & 0 & 1\\
                    0 & 0 & 1  &  0 & 0 & 0 & 0 & 0 & 0 & 1 & 0\\
                    0 & 1 & 0  &  0 & 0 & 0 & 0 & 0 & 1 & 0 & 0\\
                    0 & 1 & 1  &  0 & 0 & 0 & 0 & 1 & 0 & 0 & 0\\
                    1 & 0 & 0  &  0 & 0 & 0 & 1 & 0 & 0 & 0 & 0\\
                    1 & 0 & 1  &  0 & 0 & 1 & 0 & 0 & 0 & 0 & 0\\
                    1 & 1 & 0  &  0 & 1 & 0 & 0 & 0 & 0 & 0 & 0\\
                    1 & 1 & 1  &  1 & 0 & 0 & 0 & 0 & 0 & 0 & 0\\
                    \bottomrule
                \end{tabular}
            \end{figure}
        \item
            Schaltfunktionen:
            \begin{eqnarray*}
                y_0 = \overline{x_2} \cdot \overline{x_1} \cdot \overline{x_0} \\
                y_1 = \overline{x_2} \cdot \overline{x_1} \cdot x_0 \\
                y_2 = \overline{x_2} \cdot x_1 \cdot \overline{x_0} \\
                y_3 = \overline{x_2} \cdot x_1 \cdot x_0 \\
                y_4 = x_2 \cdot \overline{x_1} \cdot \overline{x_0} \\
                y_5 = x_2 \cdot \overline{x_1} \cdot x_0 \\
                y_6 = x_2 \cdot x_1 \cdot \overline{x_0} \\
                y_7 = x_2 \cdot x_1 \cdot x_0 \\
            \end{eqnarray*}
            Gatterschaltungen:
            \begin{figure}[H]
    \centering
        \begin{circuitikz}
        \node at(0,0) (){$x_0$};
        \node at(0,1) (){$x_1$};
        \node at(0,2) (){$x_2$};
        \node at(7,1.5) {$y_0$};
        \node[european and port] at (4,0.5) (and00){};
        \node[european and port] at (6,1.5) (and01){};
        \node[european not port] at (2,0) (not00){};
        \node[european not port] at (2,1) (not01){};
        \node[european not port] at (2,2) (not02){};
        \draw (and00.out) -| (and01.in 2);
        \draw (not00.out) -| (and00.in 2);
        \draw (not01.out) -| (and00.in 1);
        \draw (not02.out) -| (and01.in 1);
    \end{circuitikz}
\end{figure}

\begin{figure}[H]
    \centering
        \begin{circuitikz}
        \node at(0,0) (){$x_0$};
        \node at(0,1) (){$x_1$};
        \node at(0,2) (){$x_2$};
        \node at(7,1.5) {$y_1$};
        \node[european and port] at (4,0.5) (and00){};
        \node[european and port] at (6,1.5) (and01){};
        \node[european not port] at (2,0) (not00){};
        \node[european not port] at (2,1) (not01){};
        \draw (and00.out) -| (and01.in 2);
        \draw (not00.out) -| (and00.in 2);
        \draw (not01.out) -| (and00.in 1);
        \draw (0.5,2) -| (and01.in 1);
    \end{circuitikz}
\end{figure}

\begin{figure}[H]
    \centering
        \begin{circuitikz}
        \node at(0,0) (){$x_0$};
        \node at(0,1) (){$x_1$};
        \node at(0,2) (){$x_2$};
        \node at(7,1.5) {$y_2$};
        \node[european and port] at (4,0.5) (and00){};
        \node[european and port] at (6,1.5) (and01){};
        \node[european not port] at (2,0) (not00){};
        \node[european not port] at (2,2) (not02){};
        \draw (and00.out) -| (and01.in 2);
        \draw (not00.out) -| (and00.in 2);
        \draw (0.5,1) -| (and00.in 1);
        \draw (not02.out) -| (and01.in 1);
    \end{circuitikz}
\end{figure}

\begin{figure}[H]
    \centering
        \begin{circuitikz}
        \node at(0,0) (){$x_0$};
        \node at(0,1) (){$x_1$};
        \node at(0,2) (){$x_2$};
        \node at(7,1.5) {$y_3$};
        \node[european and port] at (4,0.5) (and00){};
        \node[european and port] at (6,1.5) (and01){};
        \node[european not port] at (2,0) (not00){};
        \draw (and00.out) -| (and01.in 2);
        \draw (not00.out) -| (and00.in 2);
        \draw (0.5,1) -| (and00.in 1);
        \draw (0.5,2) -| (and01.in 1);
    \end{circuitikz}
\end{figure}

\begin{figure}[H]
    \centering
        \begin{circuitikz}
        \node at(0,0) (){$x_0$};
        \node at(0,1) (){$x_1$};
        \node at(0,2) (){$x_2$};
        \node at(7,1.5) {$y_4$};
        \node[european and port] at (4,0.5) (and00){};
        \node[european and port] at (6,1.5) (and01){};
        \node[european not port] at (2,1) (not01){};
        \node[european not port] at (2,2) (not02){};
        \draw (and00.out) -| (and01.in 2);
        \draw (0.5,0) -| (and00.in 2);
        \draw (not01.out) -| (and00.in 1);
        \draw (not02.out) -| (and01.in 1);
    \end{circuitikz}
\end{figure}

\begin{figure}[H]
    \centering
        \begin{circuitikz}
        \node at(0,0) (){$x_0$};
        \node at(0,1) (){$x_1$};
        \node at(0,2) (){$x_2$};
        \node at(7,1.5) {$y_5$};
        \node[european and port] at (4,0.5) (and00){};
        \node[european and port] at (6,1.5) (and01){};
        \node[european not port] at (2,1) (not01){};
        \draw (and00.out) -| (and01.in 2);
        \draw (0.5,0) -| (and00.in 2);
        \draw (not01.out) -| (and00.in 1);
        \draw (0.5,2) -| (and01.in 1);
    \end{circuitikz}
\end{figure}

\begin{figure}[H]
    \centering
        \begin{circuitikz}
        \node at(0,0) (){$x_0$};
        \node at(0,1) (){$x_1$};
        \node at(0,2) (){$x_2$};
        \node at(7,1.5) {$y_6$};
        \node[european and port] at (4,0.5) (and00){};
        \node[european and port] at (6,1.5) (and01){};
        \node[european not port] at (2,2) (not02){};
        \draw (and00.out) -| (and01.in 2);
        \draw (0.5,0) -| (and00.in 2);
        \draw (0.5,1) -| (and00.in 1);
        \draw (not02.out) -| (and01.in 1);
    \end{circuitikz}
\end{figure}

\begin{figure}[H]
    \centering
        \begin{circuitikz}
        \node at(0,0) (){$x_0$};
        \node at(0,1) (){$x_1$};
        \node at(0,2) (){$x_2$};
        \node at(7,1.5) {$y_7$};
        \node[european and port] at (4,0.5) (and00){};
        \node[european and port] at (6,1.5) (and01){};
        \draw (and00.out) -| (and01.in 2);
        \draw (0.5,0) -| (and00.in 2);
        \draw (0.5,1) -| (and00.in 1);
        \draw (0.5,2) -| (and01.in 1);
    \end{circuitikz}
\end{figure}
 % You don't really want to see this
    \end{enumerate}

    \subsection{}
    \begin{enumerate}[label=(\alph*)]
        \item 
            Gray-Codierung:
            \begin{figure}[H]
                \centering
                \begin{tabular}{ccc|ccc}
                    \toprule
                    $x_2$ & $x_1$ & $x_0$ & $y_2$ & $y_1$ & $y_0$ \\
                    \midrule
                    0 & 0 & 0 & 0 & 0 & 0\\
                    0 & 0 & 1 & 0 & 0 & 1\\
                    0 & 1 & 0 & 0 & 1 & 1\\
                    0 & 1 & 1 & 1 & 1 & 1\\
                    1 & 0 & 0 & 1 & 0 & 1\\
                    1 & 0 & 1 & 1 & 0 & 0\\
                    1 & 1 & 0 & 1 & 1 & 0\\
                    1 & 1 & 1 & 0 & 1 & 0\\
                    \bottomrule
                \end{tabular}
            \end{figure}
        \item TODO
        \item Wenn Grey-Codierte Daten über einen Kanal übertragen werden und
            es zu Fehlern kommt (zum Beispiel durch Rauschen oder ungenaue Messungen), so ist das decodierte Codewort in fast allen Fällen nur um ein
            Bit gegenüber dem gesendeten Codewort falsch. Bei \glqq{}normaler\grqq{} Codierung treten Fehler in Codewörtern, die mehrere Bits des decodierten Codeworts betreffen, auch schon bei kleinen Fehlern auf.
			Im konkreten Beispiel der angegebenen Codescheibe kann es passieren, dass die Scheibe um einen falschen Winkel gedreht wird. Nehmen wir an, eine 3 (also in binär 011) wird \glqq{}gesendet\grqq{}, jedoch eine 4 \glqq{}empfangen\grqq{}. Der Fehler beträgt dann 3 Bit. Bei einer Gray-Codierung kann der selbe Winkelfehler maximal einen Datenfehler von einem Bit bedeuten.
        \item 
            Ausgehend von der DKNF mit Quine-McCluskey:
            \begin{eqnarray*}
                y_0(x_0, x_1, _2) &=& 
                \overline{x_2} \cdot \overline{x_1} \cdot x_0 + 
                \overline{x_2} \cdot x_1 \cdot \overline{x_0} +
                \overline{x_2} \cdot x_1 \cdot x_0 +
                x_2 \cdot \overline{x_1} \cdot \overline{x_0} \\
                &=& (\overline{x_2} \cdot \overline{x_1} \cdot x_0 + 
                \overline{x_2} \cdot x_1 \cdot x_0) +
                (\overline{x_2} \cdot x_1 \cdot \overline{x_0} +
                x_2 \cdot \overline{x_1} \cdot \overline{x_0}) \\
                &=& \overline{x_2} \cdot x_0 +
                \overline{x_0} \cdot (x_2 \cdot \overline{x_1} + \overline{x_2} \cdot x_1) \\
                &=& \overline{x_2} \cdot x_0 + \overline{x_0} \cdot (x_1 \oplus x_2)
            \end{eqnarray*}
    \end{enumerate}

    \subsection{}
    \begin{enumerate}[label=(\alph*)]
        \item Wertetabelle:
            \begin{figure}[H]
                \centering
                \begin{tabular}{c|cccc|cccc}
                    \toprule
                    Dezimal & \multicolumn{4}{c|}{Binär} & \multicolumn{4}{c}{Aiken-Code} \\
                    $d$ & $x_3$ & $x_2$ & $x_1$ & $x_0$ & $y_3$ & $y_2$ & $y_1$ & $y_0$ \\
                    \midrule
					0 & 0 & 0 & 0 & 0 & 0 & 0 & 0 & 0 \\
					1 & 0 & 0 & 0 & 1 & 0 & 0 & 0 & 1 \\
					2 & 0 & 0 & 1 & 0 & 0 & 0 & 1 & 0 \\
					3 & 0 & 0 & 1 & 1 & 0 & 0 & 1 & 1 \\
					4 & 0 & 1 & 0 & 0 & 0 & 1 & 0 & 0 \\
					5 & 0 & 1 & 0 & 1 & 1 & 0 & 1 & 1 \\
					6 & 0 & 1 & 1 & 0 & 1 & 1 & 0 & 0 \\
					7 & 0 & 1 & 1 & 1 & 1 & 1 & 0 & 1 \\
					8 & 1 & 0 & 0 & 0 & 1 & 1 & 1 & 0 \\
					9 & 1 & 0 & 0 & 1 & 1 & 1 & 1 & 1 \\
					10 & 1 & 0 & 1 & 0 & - & - & - & - \\
					11 & 1 & 0 & 1 & 1 & - & - & - & - \\
					12 & 1 & 1 & 0 & 0 & - & - & - & - \\
					13 & 1 & 1 & 0 & 1 & - & - & - & - \\
					14 & 1 & 1 & 1 & 0 & - & - & - & - \\
					15 & 1 & 1 & 1 & 1 & - & - & - & - \\
                    \bottomrule
                \end{tabular}
            \end{figure}
		Die Zahlen 10 bis 15 sind nicht mit einem 4-Bit Aiken-Code darstellbar.
		\item 
		\begin{eqnarray*}
			y_0 &=& \minTerm{1}{0}{0}{0} + \minTerm{1}{1}{0}{0} + \minTerm{1}{0}{1}{0} + \minTerm{1}{1}{1}{0} + \minTerm{1}{0}{0}{1} \\
			y_1 &=& \minTerm{0}{1}{0}{0} + \minTerm{1}{1}{0}{0} + \minTerm{1}{0}{1}{0} + \minTerm{0}{0}{0}{1} + \minTerm{1}{0}{0}{1} \\
			y_2 &=& \minTerm{0}{0}{1}{0} + \minTerm{0}{1}{1}{0} + \minTerm{1}{1}{1}{0} + \minTerm{0}{0}{0}{1} + \minTerm{1}{0}{0}{1} \\
			y_3 &=& \minTerm{1}{0}{1}{0} + \minTerm{0}{1}{1}{0} + \minTerm{1}{1}{1}{0} + \minTerm{0}{0}{0}{1} + \minTerm{1}{0}{0}{1} \\
		\end{eqnarray*}
		\item 
			KV-Diagramm für $y_0$:
			\begin{figure}[H]
                \centering
                \begin{tabular}{cc|cccc|cc}
                    & &  & \multicolumn{2}{c}{$x_0$} & & \\
                    & & 0 & 1 & 1 & 0\\
                    \midrule
                    \multirow{4}{*}{$x_1$} & 0 & 0 & \cellcolor{blue!50}1 & \cellcolor{blue!50}1 & 0 & 0 &\multirow{4}{*}{$x_3$}\\
                     & 1 & 0 & \cellcolor{blue!50}1 & \cellcolor{blue!50}1 & 0 & 0\\
                     & 1 & X & \cellcolor{blue!50}X & \cellcolor{blue!50}X & X & 1\\
                     & 0 & 0 & \cellcolor{blue!50}1 & \cellcolor{blue!50}X & X & 1\\
                    \midrule
                    & & 0 & 0 & 1 & 1\\
                    & &  & \multicolumn{2}{c}{$x_2$} & & \\
                \end{tabular}
            \end{figure}
			Daraus folgt:
			\begin{equation*}
				y_0 = x_0
			\end{equation*}

			KV-Diagramm für $y_1$:
			\begin{figure}[H]
                \centering
                \begin{tabular}{cc|cccc|cc}
                    & &  & \multicolumn{2}{c}{$x_0$} & & \\
                    & & 0 & 1 & 1 & 0\\
                    \midrule
                    \multirow{4}{*}{$x_1$} & 0 & 0 & 0 & \cellcolor{green!50}1 & 0 & 0 &\multirow{4}{*}{$x_3$}\\
                     & 1 & \cellcolor{blue!50}1 & \cellcolor{blue!50}1 & 0 & 0 & 0\\
                     & 1 & \cellcolor{purple!50}X & \cellcolor{purple!50}X & X & X & 1\\
                     & 0 & \cellcolor{red!50}1 & \cellcolor{red!50}1 & \cellcolor{green!50}X & X & 1\\
                    \midrule
                    & & 0 & 0 & 1 & 1\\
                    & &  & \multicolumn{2}{c}{$x_2$} & & \\
                \end{tabular}
            \end{figure}
			Daraus folgt:
			\begin{equation*}
				y_1 = \textcolor{blue}{x_1 \cdot \overline{x_2}} + \textcolor{red}{\overline{x_2} \cdot x_3} + \textcolor{green}{x_0 \cdot \overline{x_1} \cdot x_2}
			\end{equation*}

			KV-Diagramm für $y_2$:
			\begin{figure}[H]
                \centering
                \begin{tabular}{cc|cccc|cc}
                    & &  & \multicolumn{2}{c}{$x_0$} & & \\
                    & & 0 & 1 & 1 & 0\\
                    \midrule
                    \multirow{4}{*}{$x_1$} & 0 & 0 & 0 & 0 & \cellcolor{green!50}1 & 0 &\multirow{4}{*}{$x_3$}\\
                     & 1 & 0 & 0 & \cellcolor{red!50}1 & \cellcolor{yellow!50}1 & 0\\
                     & 1 & \cellcolor{blue!50}X & \cellcolor{blue!50}X & \cellcolor{purple!50}X & \cellcolor{purple!50}X & 1\\
                     & 0 & \cellcolor{blue!50}1 & \cellcolor{blue!50}1 & \cellcolor{blue!50}X & \cellcolor{blue!50}X & 1\\
                    \midrule
                    & & 0 & 0 & 1 & 1\\
                    & &  & \multicolumn{2}{c}{$x_2$} & & \\
                \end{tabular}
            \end{figure}
			Daraus folgt:
			\begin{equation*}
				y_2 = \textcolor{blue}{x_3} + \textcolor{red}{x_1 \cdot x_2} + \textcolor{green}{\overline{x_0} \cdot x_2 \cdot \overline{x_3}}
			\end{equation*}

			KV-Diagramm für $y_3$:
			\begin{figure}[H]
                \centering
                \begin{tabular}{cc|cccc|cc}
                    & &  & \multicolumn{2}{c}{$x_0$} & & \\
                    & & 0 & 1 & 1 & 0\\
                    \midrule
                    \multirow{4}{*}{$x_1$} & 0 & 0 & 0 & \cellcolor{green!50}1 & 0 & 0 &\multirow{4}{*}{$x_3$}\\
                     & 1 & 0 & 0 & \cellcolor{yellow!50}1 & \cellcolor{red!50}1 & 0\\
                     & 1 & \cellcolor{blue!50}X & \cellcolor{blue!50}X & \cellcolor{purple!50}X & \cellcolor{purple!50}X & 1\\
                     & 0 & \cellcolor{blue!50}1 & \cellcolor{blue!50}1 & \cellcolor{blue!50}X & \cellcolor{blue!50}X & 1\\
                    \midrule
                    & & 0 & 0 & 1 & 1\\
                    & &  & \multicolumn{2}{c}{$x_2$} & & \\
                \end{tabular}
            \end{figure}
			Daraus folgt:
			\begin{equation*}
				y_3 = \textcolor{blue}{x_3} + \textcolor{red}{x_1 \cdot x_2} + \textcolor{green}{x_0 \cdot x_2 \cdot \overline{x_3}}
			\end{equation*}
    \end{enumerate}
\end{document}

