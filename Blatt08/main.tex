\documentclass[DIN, pagenumber=false, fontsize=11pt, parskip=half]{scrartcl}

\usepackage{amsmath}
\usepackage{amsfonts}
\usepackage{amssymb}
\usepackage{enumitem}
\usepackage[utf8]{inputenc} % this is needed for umlauts
\usepackage[ngerman]{babel} % this is needed for umlauts
\usepackage[T1]{fontenc} 
\usepackage{commath}
\usepackage{xcolor}
\usepackage{booktabs}
\usepackage{float}
\usepackage{tikz-timing}
\usepackage{tikz}
\usepackage{multirow}
\usepackage{colortbl}
\usepackage{xstring}
\usepackage{circuitikz}

\usetikzlibrary{calc,shapes.multipart,chains,arrows}

\title{Grundlagen der Rechnerarchitektur}
\author{Tim Luchterhand, Paul Nykiel (Abgabegruppe 117)}

\newcommand{\boolshitKgV}[1]{\text{kgV}(#1,30) &=& 30\\}
\newcommand{\boolshitggT}[1]{\text{ggT}(#1,30) &=& #1\\}
\newcommand{\minTerm}[4]{
    \IfEqCase{#1}{
        {0}{\overline{x_0}\ }
        {1}{x_0}
    }
    \IfEqCase{#2}{
        {0}{\overline{x_1}\ }
        {1}{x_1}
    }
    \IfEqCase{#3}{
        {0}{\overline{x_2}\ }
        {1}{x_2}
    }
    \IfEqCase{#4}{
        {0}{\overline{x_3}\ }
        {1}{x_3}
    }[\PackageError{minTerm}{Values must either be 0 or 1}]
}

\newcommand{\maxTerm}[4]{
    (
    \IfEqCase{#1}{
        {0}{\overline{x_0} + }
        {1}{x_0 + }
    }
    \IfEqCase{#2}{
        {0}{\overline{x_1} + }
        {1}{x_1 + }
    }
    \IfEqCase{#3}{
        {0}{\overline{x_2} + }
        {1}{x_2 + }
    }
    \IfEqCase{#4}{
        {0}{\overline{x_3}}
        {1}{x_3}
    }[\PackageError{minTerm}{Values must either be 0 or 1}]
    )
}

\begin{document}
    \maketitle
    \section{Umcodierer}
    \subsection{}
    \begin{enumerate}[label=(\alph*)]
        \item
            \begin{figure}[H]
                \centering
                \begin{tabular}{ccc|cccccccc}
                    \toprule
                    $x_2$ &$x_1$ & $x_0$ & $y_7$ & $y_6$ & $y_5$ & $y_4$ & $y_3$ & $y_2$ & $y_1$ & $y_0$ \\
                    \midrule
                    0 & 0 & 0  &  0 & 0 & 0 & 0 & 0 & 0 & 0 & 1\\
                    0 & 0 & 1  &  0 & 0 & 0 & 0 & 0 & 0 & 1 & 0\\
                    0 & 1 & 0  &  0 & 0 & 0 & 0 & 0 & 1 & 0 & 0\\
                    0 & 1 & 1  &  0 & 0 & 0 & 0 & 1 & 0 & 0 & 0\\
                    1 & 0 & 0  &  0 & 0 & 0 & 1 & 0 & 0 & 0 & 0\\
                    1 & 0 & 1  &  0 & 0 & 1 & 0 & 0 & 0 & 0 & 0\\
                    1 & 1 & 0  &  0 & 1 & 0 & 0 & 0 & 0 & 0 & 0\\
                    1 & 1 & 1  &  1 & 0 & 0 & 0 & 0 & 0 & 0 & 0\\
                    \bottomrule
                \end{tabular}
            \end{figure}
        \item
            Schaltfunktionen:
            \begin{eqnarray*}
                y_0 = \overline{x_2} \cdot \overline{x_1} \cdot \overline{x_0} \\
                y_1 = \overline{x_2} \cdot \overline{x_1} \cdot x_0 \\
                y_2 = \overline{x_2} \cdot x_1 \cdot \overline{x_0} \\
                y_3 = \overline{x_2} \cdot x_1 \cdot x_0 \\
                y_4 = x_2 \cdot \overline{x_1} \cdot \overline{x_0} \\
                y_5 = x_2 \cdot \overline{x_1} \cdot x_0 \\
                y_6 = x_2 \cdot x_1 \cdot \overline{x_0} \\
                y_7 = x_2 \cdot x_1 \cdot x_0 \\
            \end{eqnarray*}
            Gatterschaltungen:
            \begin{figure}[H]
    \centering
        \begin{circuitikz}
        \node at(0,0) (){$x_0$};
        \node at(0,1) (){$x_1$};
        \node at(0,2) (){$x_2$};
        \node at(7,1.5) {$y_0$};
        \node[european and port] at (4,0.5) (and00){};
        \node[european and port] at (6,1.5) (and01){};
        \node[european not port] at (2,0) (not00){};
        \node[european not port] at (2,1) (not01){};
        \node[european not port] at (2,2) (not02){};
        \draw (and00.out) -| (and01.in 2);
        \draw (not00.out) -| (and00.in 2);
        \draw (not01.out) -| (and00.in 1);
        \draw (not02.out) -| (and01.in 1);
    \end{circuitikz}
\end{figure}

\begin{figure}[H]
    \centering
        \begin{circuitikz}
        \node at(0,0) (){$x_0$};
        \node at(0,1) (){$x_1$};
        \node at(0,2) (){$x_2$};
        \node at(7,1.5) {$y_1$};
        \node[european and port] at (4,0.5) (and00){};
        \node[european and port] at (6,1.5) (and01){};
        \node[european not port] at (2,0) (not00){};
        \node[european not port] at (2,1) (not01){};
        \draw (and00.out) -| (and01.in 2);
        \draw (not00.out) -| (and00.in 2);
        \draw (not01.out) -| (and00.in 1);
        \draw (0.5,2) -| (and01.in 1);
    \end{circuitikz}
\end{figure}

\begin{figure}[H]
    \centering
        \begin{circuitikz}
        \node at(0,0) (){$x_0$};
        \node at(0,1) (){$x_1$};
        \node at(0,2) (){$x_2$};
        \node at(7,1.5) {$y_2$};
        \node[european and port] at (4,0.5) (and00){};
        \node[european and port] at (6,1.5) (and01){};
        \node[european not port] at (2,0) (not00){};
        \node[european not port] at (2,2) (not02){};
        \draw (and00.out) -| (and01.in 2);
        \draw (not00.out) -| (and00.in 2);
        \draw (0.5,1) -| (and00.in 1);
        \draw (not02.out) -| (and01.in 1);
    \end{circuitikz}
\end{figure}

\begin{figure}[H]
    \centering
        \begin{circuitikz}
        \node at(0,0) (){$x_0$};
        \node at(0,1) (){$x_1$};
        \node at(0,2) (){$x_2$};
        \node at(7,1.5) {$y_3$};
        \node[european and port] at (4,0.5) (and00){};
        \node[european and port] at (6,1.5) (and01){};
        \node[european not port] at (2,0) (not00){};
        \draw (and00.out) -| (and01.in 2);
        \draw (not00.out) -| (and00.in 2);
        \draw (0.5,1) -| (and00.in 1);
        \draw (0.5,2) -| (and01.in 1);
    \end{circuitikz}
\end{figure}

\begin{figure}[H]
    \centering
        \begin{circuitikz}
        \node at(0,0) (){$x_0$};
        \node at(0,1) (){$x_1$};
        \node at(0,2) (){$x_2$};
        \node at(7,1.5) {$y_4$};
        \node[european and port] at (4,0.5) (and00){};
        \node[european and port] at (6,1.5) (and01){};
        \node[european not port] at (2,1) (not01){};
        \node[european not port] at (2,2) (not02){};
        \draw (and00.out) -| (and01.in 2);
        \draw (0.5,0) -| (and00.in 2);
        \draw (not01.out) -| (and00.in 1);
        \draw (not02.out) -| (and01.in 1);
    \end{circuitikz}
\end{figure}

\begin{figure}[H]
    \centering
        \begin{circuitikz}
        \node at(0,0) (){$x_0$};
        \node at(0,1) (){$x_1$};
        \node at(0,2) (){$x_2$};
        \node at(7,1.5) {$y_5$};
        \node[european and port] at (4,0.5) (and00){};
        \node[european and port] at (6,1.5) (and01){};
        \node[european not port] at (2,1) (not01){};
        \draw (and00.out) -| (and01.in 2);
        \draw (0.5,0) -| (and00.in 2);
        \draw (not01.out) -| (and00.in 1);
        \draw (0.5,2) -| (and01.in 1);
    \end{circuitikz}
\end{figure}

\begin{figure}[H]
    \centering
        \begin{circuitikz}
        \node at(0,0) (){$x_0$};
        \node at(0,1) (){$x_1$};
        \node at(0,2) (){$x_2$};
        \node at(7,1.5) {$y_6$};
        \node[european and port] at (4,0.5) (and00){};
        \node[european and port] at (6,1.5) (and01){};
        \node[european not port] at (2,2) (not02){};
        \draw (and00.out) -| (and01.in 2);
        \draw (0.5,0) -| (and00.in 2);
        \draw (0.5,1) -| (and00.in 1);
        \draw (not02.out) -| (and01.in 1);
    \end{circuitikz}
\end{figure}

\begin{figure}[H]
    \centering
        \begin{circuitikz}
        \node at(0,0) (){$x_0$};
        \node at(0,1) (){$x_1$};
        \node at(0,2) (){$x_2$};
        \node at(7,1.5) {$y_7$};
        \node[european and port] at (4,0.5) (and00){};
        \node[european and port] at (6,1.5) (and01){};
        \draw (and00.out) -| (and01.in 2);
        \draw (0.5,0) -| (and00.in 2);
        \draw (0.5,1) -| (and00.in 1);
        \draw (0.5,2) -| (and01.in 1);
    \end{circuitikz}
\end{figure}
 % You don't really want to see this
    \end{enumerate}
\end{document}
