\documentclass[DIN, pagenumber=false, fontsize=11pt, parskip=half]{scrartcl}

\usepackage{amsmath}
\usepackage{amsfonts}
\usepackage{amssymb}
\usepackage{enumitem}
\usepackage[utf8]{inputenc} % this is needed for umlauts
\usepackage[ngerman]{babel} % this is needed for umlauts
\usepackage[T1]{fontenc} 
\usepackage{commath}
\usepackage{xcolor}
\usepackage{booktabs}
\usepackage{float}
\usepackage{tikz-timing}
\usepackage{tikz}
\usepackage{multirow}
\usepackage{colortbl}
\usepackage{xstring}

\usetikzlibrary{calc,shapes.multipart,chains,arrows}

\title{Grundlagen der Rechnerarchitektur}
\author{Tim Luchterhand, Paul Nykiel (Abgabegruppe 117)}

\newcommand{\boolshitKgV}[1]{\text{kgV}(#1,30) &=& 30\\}
\newcommand{\boolshitggT}[1]{\text{ggT}(#1,30) &=& #1\\}
\newcommand{\minTerm}[4]{
    \IfEqCase{#1}{
        {0}{\overline{x_0}\ }
        {1}{x_0}
    }
    \IfEqCase{#2}{
        {0}{\overline{x_1}\ }
        {1}{x_1}
    }
    \IfEqCase{#3}{
        {0}{\overline{x_2}\ }
        {1}{x_2}
    }
    \IfEqCase{#4}{
        {0}{\overline{x_3}\ }
        {1}{x_3}
    }[\PackageError{minTerm}{Values must either be 0 or 1}]
}

\newcommand{\maxTerm}[4]{
    (
    \IfEqCase{#1}{
        {0}{\overline{x_0} + }
        {1}{x_0 + }
    }
    \IfEqCase{#2}{
        {0}{\overline{x_1} + }
        {1}{x_1 + }
    }
    \IfEqCase{#3}{
        {0}{\overline{x_2} + }
        {1}{x_2 + }
    }
    \IfEqCase{#4}{
        {0}{\overline{x_3}}
        {1}{x_3}
    }[\PackageError{minTerm}{Values must either be 0 or 1}]
    )
}

\definecolor{mixtureColor}{HTML}{555439}

\begin{document}
    \maketitle
    \textbf{Hinweis: } Jede Zahl, bei der keine Basis spezifiziert ist, ist im 10er System zu interpretieren, sofern nicht anders angegeben.
    \section{Schaltalgebra}
    \subsection{}
    \begin{enumerate}[label = (\alph*)]
        \item Die Schaltalgebra ist eine Teilmenge der Boolschen Algebra mit einer zweiwertigen Trägermenge. 
            Das heißt es gibt nur zwei möglichen Werte, anstatt beliebig vielen Werten bei einer Boolschen Algebra.
        \item 
            \begin{enumerate}[label = (\roman*)]
                \item 
                    Nur mit NAND-Gattern:
                    \begin{equation*}
                        x_1 \cdot x_2 \stackrel{\text{P7}}{=} \overline{\overline{x_1 \cdot x_2}} \stackrel{\text{P3}}{=} \overline{\overline{x_1 \cdot x_2} \cdot \overline{x_1 \cdot x_2}}
                    \end{equation*}
                    Nur mit NOR-Gattern:
                    \begin{equation*}
                        x_1 \cdot x_2 \stackrel{\text{P3}}{=} \overline{\overline{x_1 \cdot x_2}} \stackrel{\text{P8}}{=} \overline{\overline{x_1} + \overline{x_2}} \stackrel{\text{P5}}{=} \overline{\overline{x_1 + 0} + \overline{x_2 + 0}}
                    \end{equation*}
                \item
                    Nur mit NAND-Gattern:
                    \begin{equation*}
                        x_1 \cdot \overline{x_2} + \overline{x_1} \cdot x_2 \stackrel{\text{P3}}{=} x_1 \cdot \overline{x_2 \cdot x_2} + \overline{x_1 \cdot x_1} \cdot x_2  
                        \stackrel{\text{P7}}{=} \overline{\overline{x_1 \cdot \overline{x_2 \cdot x_2} + \overline{x_1 \cdot x_1} \cdot x_2}}
                        \stackrel{\text{P8'}}{=} \overline{\overline{x_1 \cdot \overline{x_2 \cdot x_2}} \cdot \overline{ \overline{x_1 \cdot x_1} \cdot x_2}}
                    \end{equation*}
                    Nur mit NOR-Gattern:
                    \begin{eqnarray*}
                        x_1 \cdot \overline{x_2} + \overline{x_1} \cdot x_2 &\stackrel{\text{1b}}{=}& \overline{\overline{x_1 + 0} + \overline{\overline{x_2} + 0}}
                        + \overline{\overline{\overline{x_1} + 0} + \overline{x_2 + 0}} \\
                        &\stackrel{\text{P5', P7}}{=}& \overline{\overline{x_1 + 0} + x_2} + \overline{x_1 + \overline{x_2 + 0}} \\
                        &\stackrel{\text{P5', P7}}{=}& \overline{\overline{\overline{\overline{x_1 + 0} + x_2} + \overline{x_1 + \overline{x_2 + 0}}} + 0}
                    \end{eqnarray*}
            \end{enumerate}
    \end{enumerate}

    \subsection{}
    \begin{enumerate}[label = (\alph*)]
        \item 
             Wertetabelle

            \begin{tabular}{l | l | l | l | l}
                \hline
                $2^3 = x_3$ & $2^2 = x_2$ & $2^1 = x_1$ & $2^0 = x_0$ & $f(x)$ \\ \hline
                0 & 0 & 0 & 0 & 0 \\ \hline
                0 & 0 & 0 & 1 & 1 \\ \hline
                0 & 0 & 1 & 0 & 1 \\ \hline
                0 & 0 & 1 & 1 & 1 \\ \hline
                0 & 1 & 0 & 0 & 1 \\ \hline
                0 & 1 & 0 & 1 & 1 \\ \hline
                0 & 1 & 1 & 0 & 1 \\ \hline
                0 & 1 & 1 & 1 & 1 \\ \hline
                1 & 0 & 0 & 0 & 1 \\ \hline
                1 & 0 & 0 & 1 & 0 \\ \hline
                1 & 0 & 1 & 0 & 0 \\ \hline
                1 & 0 & 1 & 1 & 1 \\ \hline
                1 & 1 & 0 & 0 & 0 \\ \hline
                1 & 1 & 0 & 1 & 1 \\ \hline
                1 & 1 & 1 & 0 & 0 \\ \hline
                1 & 1 & 1 & 1 & 0 \\ \hline
            \end{tabular}
        \item
            \begin{eqnarray*}
                f_{\text{DKNF}} &=& \minTerm{1}{0}{0}{0} + \minTerm{0}{1}{0}{0} + \minTerm{1}{1}{0}{0} + \minTerm{0}{0}{1}{0} \\
                &+& \minTerm{1}{0}{1}{0} + \minTerm{0}{1}{1}{0} + \minTerm{1}{1}{1}{0} + \minTerm{0}{0}{0}{1} \\
                &+& \minTerm{1}{1}{0}{1} + \minTerm{1}{0}{1}{1}
            \end{eqnarray*}
            \begin{eqnarray*}
                f_{\text{KKNF}} &=& \maxTerm{1}{1}{1}{1} \cdot \maxTerm{0}{1}{1}{0} \cdot \maxTerm{1}{0}{1}{0} \cdot \maxTerm{1}{1}{0}{0} \\
                &\cdot& \maxTerm{1}{0}{0}{0} \cdot \maxTerm{0}{0}{0}{0}
            \end{eqnarray*}
        \item 
            Da DKNF und KKNF äquivalent sind wird auf Basis der Wahrheitstabelle minimiert.

            Karnaugh-Veitch-Diagramm:
            \begin{figure}[H]
                \centering
                \begin{tabular}{cc|cccc|cc}
                    & &  & \multicolumn{2}{c}{$x_1$} & & \\
                    & & 0 & 1 & 1 & 0\\
                    \midrule
                    \multirow{4}{*}{$x_2$} & 0 & 0 & \cellcolor{blue!25}1 & \cellcolor{purple!25}1 & \cellcolor{red!25}1 & 0 &\multirow{4}{*}{$x_4$}\\
                     & 1 & \cellcolor{green!25}1 & \cellcolor{gray!25}1 & \cellcolor{red!25}1 & \cellcolor{red!25}1 & 0\\
                     & 1 & 0 & \cellcolor{yellow!25}1 & 0 & 0 & 1\\
                     & 0 & \cellcolor{cyan!25}1 & 0 & \cellcolor{purple!25}1 & 0 & 1\\
                    \midrule
                    & & 0 & 0 & 1 & 1\\
                    & &  & \multicolumn{2}{c}{$x_3$} & & \\
                \end{tabular}
            \end{figure}
            Daraus ergibt sich:
            \begin{equation*}
                f_\text{Min}(x) = \textcolor{red}{\overline{x_4}\cdot x_3} + 
                \textcolor{blue}{\overline{x_4}\cdot x_1 \cdot \overline{x_3}} + 
                \textcolor{green}{\overline{x_3} \cdot x_2 \cdot \overline{x_4}}+
                \textcolor{yellow}{x_2 \cdot \overline{x_3} \cdot x_1}+
                \textcolor{purple}{x_1 \cdot \overline{x_2} \cdot x_3}+
                \textcolor{cyan}{\overline{x_1} \cdot \overline{x_2} \cdot \overline{x_3} \cdot x_4}
            \end{equation*}
    \end{enumerate}


    \subsection{}
    \begin{enumerate}[label=(\alph*)]
        \item
            DKNF:
            \begin{eqnarray*}
                f_\text{DKNF}(x) &=& \overline{x_1} \cdot \overline{x_2} \cdot x_3 \\
                &&+ \overline{x_1} \cdot x_2 \cdot \overline{x_3} \\
                &&+ \overline{x_1} \cdot x_2 \cdot x_3 \\
                &&+ x_1 \cdot \overline{x_2} \cdot \overline{x_3} \\
                &&+ x_1 \cdot x_2 \cdot x_3
            \end{eqnarray*}
            KKNF:
            \begin{eqnarray*}
                f_\text{KKNF}(x) &=& (x_1 + x_2 + x_3) \\
                &&\cdot (\overline{x_1} + x_2 + \overline{x_3})\\
                &&\cdot (\overline{x_1} + \overline{x_2} + x_3)
            \end{eqnarray*}
        \item
            \begin{eqnarray*}
                f_\text{DKNF}(x) &=& \overline{x_1} \cdot \overline{x_2} \cdot x_3 \\
                &&+ \overline{x_1} \cdot x_2 \cdot \overline{x_3} \\
                &&+ \overline{x_1} \cdot x_2 \cdot x_3 \\
                &&+ x_1 \cdot \overline{x_2} \cdot \overline{x_3} \\
                &&+ x_1 \cdot x_2 \cdot x_3\\
                &=& \overline{x_1} \cdot (\overline{x_2} \cdot x_3 + x_2 \cdot \overline{x_3} + x_2 \cdot x_3)\\
                &&+ x_1 \cdot (x_2 \cdot x_3 + \overline{x_2} \cdot \overline{x_3})\\
                &=& \overline{x_1} \cdot (x_2 + x_3) + x_1 \cdot (x_2 \cdot x_3 + \overline{x_2} \cdot \overline{x_3})\\
                &=& \overline{\overline{\overline{x_1} \cdot (x_2 + x_3) + x_1 \cdot (x_2 \cdot x_3 + \overline{x_2} \cdot \overline{x_3})}}\\
                &=& \overline{\overline{(\overline{x_1} \cdot (x_2 + x_3))} \cdot \overline{(x_1 \cdot (x_2 \cdot x_3 + \overline{x_2} \cdot \overline{x_3}))}}\\
                &=& \overline{(x_1 + \overline{(x_2 + x_3)}) \cdot (\overline{x_1} + \overline{(x_2 \cdot x_3 + \overline{x_2} \cdot \overline{x_3})})}\\
                &=& \overline{(x_1 + (\overline{x_2} \cdot \overline{x_3})) \cdot (\overline{x_1} + (\overline{(x_2 \cdot x_3)} \cdot \overline{(\overline{x_2} \cdot \overline{x_3})}))}\\
                &=& \overline{(x_1 + (\overline{x_2} \cdot \overline{x_3})) \cdot (\overline{x_1} + ((\overline{x_2} + \overline{x_3}) \cdot (x_2 + x_3)))}\\
                &=& \overline{(x_1 + (\overline{x_2} \cdot \overline{x_3})) \cdot (\overline{x_1} + (x_2 \cdot \overline{x_2} + x_2 \cdot \overline{x_3} + x_3 \cdot \overline{x_2} + x_3 \cdot \overline{x_3}))}\\
                &=& \overline{(x_1 + (\overline{x_2} \cdot \overline{x_3})) \cdot (\overline{x_1} + x_2 \cdot \overline{x_3} + x_3 \cdot \overline{x_2})}\\
                &=& \overline{x_1 \cdot \overline{x_1} + x_1 \cdot x_2 \cdot \overline{x_3} + x_1 \cdot \overline{x_2} \cdot x_3} \\
                &&  \overline{+ \overline{x_1} \cdot \overline{x_2} \cdot \overline{x_3}
                    + x_2 \cdot \overline{x_3} \cdot \overline{x_2} \cdot \overline{x_3}
                    + \overline{x_2} \cdot x_3 \cdot \overline{x_2} \cdot \overline{x_3}}\\
                &=& \overline{x_1 \cdot x_2 \cdot \overline{x_3} + x_1 \cdot \overline{x_2} \cdot x_3 
                    + \overline{x_1} \cdot \overline{x_2} \cdot \overline{x_3}}\\
                &=& \overline{x_1 \cdot x_2 \cdot \overline{x_3}} \cdot \overline{x_1 \cdot \overline{x_2} \cdot x_3}
                    \cdot \overline{\overline{x_1} \cdot \overline{x_2} \cdot \overline{x_3}}\\
                &=& (\overline{x_1} + \overline{x_2} + x_3) \cdot (\overline{x_1} + x_2 + \overline{x_3})
                    \cdot (x_1 + x_2 + x_3)\\
                &=& f_\text{KKNF}(x)
            \end{eqnarray*}
        \item 
            Karnaugh-Veitch-Diagramm:
            \begin{figure}[H]
                \centering
                \begin{tabular}{cc|cccc}
                    & &  & \multicolumn{2}{c}{$x_2$}\\
                    & & 1 & 1 & 0 & 0\\
                    \midrule
                    \multirow{ 2}{*}{$x_1$} & 0 & \cellcolor{blue!25}1 & \cellcolor{mixtureColor!25}1 & \cellcolor{red!25}1 & 0\\
                     & 1 & 0 & \cellcolor{green!25}1 & 0 & \cellcolor{yellow!25}1\\
                    \midrule
                    & & 0 & 1 & 1 & 0\\
                    & &  & \multicolumn{2}{c}{$x_3$}\\
                \end{tabular}
            \end{figure}
            Daraus ergibt sich:
            \begin{equation*}
                f_\text{Min}(x) = \textcolor{blue}{\overline{x_1}\cdot x_2} + 
                \textcolor{green}{x_2 \cdot x_3} + 
                \textcolor{red}{x_3 \cdot \overline{x_1}}+
                \textcolor{yellow}{x_1 \cdot \overline{x_2} \cdot \overline{x_3}}
            \end{equation*}
        \item
            DKNF:
            \begin{eqnarray*}
                g_\text{DKNF}(x) &=& (\overline{x_1} \cdot \overline{x_2} \cdot x_3)\\
                &&+ (\overline{x_1} \cdot x_2 \cdot \overline{x_3})\\
                &&+ (x_1 \cdot \overline{x_2} \cdot \overline{x_3})\\
                &&+ (x_1 \cdot \overline{x_2} \cdot x_3)\\
                &&+ (x_1 \cdot x_2  \cdot \overline{x_3})\\
                &&+ (x_1 \cdot x_2 \cdot x_3)
            \end{eqnarray*}

            KKNF:
            \begin{eqnarray*}
                g_\text{KKNF}(x) &=& (x_1 + \overline{x_2} + \overline{x_3})\\
                &&\cdot (x_1 + x_2 + x_3)
            \end{eqnarray*}
        \item
            \begin{eqnarray*}
                g_\text{DKNF}(x) &=& (\overline{x_1} \cdot \overline{x_2} \cdot x_3)\\
                &&+ (\overline{x_1} \cdot x_2 \cdot \overline{x_3})\\
                &&+ (x_1 \cdot \overline{x_2} \cdot \overline{x_3})\\
                &&+ (x_1 \cdot \overline{x_2} \cdot x_3)\\
                &&+ (x_1 \cdot x_2  \cdot \overline{x_3})\\
                &&+ (x_1 \cdot x_2 \cdot x_3)\\
                &=& \overline{x_1} \cdot (\overline{x_2} \cdot x_3 + x_2 \cdot \overline{x_3})\\
                &&+ x_1 \cdot (\overline{x_2} \cdot \overline{x_3} + \overline{x_2} \cdot x_3 + x_2 \cdot \overline{x_3} + \overline{x_2} \cdot \overline{x_3})\\
                &=& \overline{x_1} \cdot (\overline{x_2} \cdot x_3 + x_2 \cdot \overline{x_3}) + x_1\\
                &=& \overline{\overline{\overline{x_1} \cdot (\overline{x_2} \cdot x_3 + x_2 \cdot \overline{x_3}) + x_1}}\\
                &=& \overline{\overline{(\overline{x_1} \cdot (\overline{x_2} \cdot x_3 + x_2 \cdot \overline{x_3}))} \cdot \overline{x_1}} \\
                &=& \overline{(x_1 + \overline{(\overline{x_2} \cdot x_3 + x_2 \cdot \overline{x_3})}) \cdot \overline{x_1}} \\
                &=& \overline{(x_1 + (\overline{\overline{x_2} \cdot x_3} \cdot \overline{x_2 \cdot \overline{x_3}})) \cdot \overline{x_1}}\\
                &=& \overline{(x_1 + (x_2 + \overline{x_3}) \cdot (\overline{x_2} + x_3)) \cdot \overline{x_1}}\\
                &=& \overline{x_1 \cdot \overline {x_1} + (x_2 + \overline{x_3}) \cdot (\overline{x_2} + x_3) \cdot \overline{x_1}}\\
                &=& \overline{(x_2 + \overline{x_3}) \cdot (\overline{x_2} + x_3) \cdot \overline{x_1}}\\
                &=& \overline{(x_2 \cdot \overline{x_2} + x_2 \cdot x_3 + \overline{x_2} \cdot \overline{x_3} + x_3 \cdot \overline{x_3}) \cdot \overline{x_1}}\\
                &=& \overline{(x_2 \cdot x_3 + \overline{x_2} \cdot \overline{x_3}) \cdot \overline{x_1}}\\
                &=& \overline{(x_2 \cdot x_3 \cdot \overline{x_1}) + (\overline{x_2} \cdot \overline{x_3} \cdot \overline{x_1})}\\
                &=& \overline{(x_2 \cdot x_3 \cdot \overline{x_1})} \cdot \overline{(\overline{x_2} \cdot \overline{x_3} \cdot \overline{x_1})}\\
                &=& (\overline{x_2} + \overline{x_3} + x_1) \cdot (x_2 + x_3 + x_1)\\
                &=& g_\text{KKNF}(x)
            \end{eqnarray*}
        \item 
            Karnaugh-Veitch-Diagramm:
            \begin{figure}[H]
                \centering
                \begin{tabular}{cc|cccc}
                    & &  & \multicolumn{2}{c}{$x_2$}\\
                    & & 1 & 1 & 0 & 0\\
                    \midrule
                    \multirow{ 2}{*}{$x_1$} & 0 & \cellcolor{green!25}1& 0 & \cellcolor{red!25}1 & 0\\
                     & 1 & \cellcolor{blue!25}1 & \cellcolor{blue!25}1 & \cellcolor{blue!25}1 & \cellcolor{blue!25}1\\
                    \midrule
                    & & 0 & 1 & 1 & 0\\
                    & &  & \multicolumn{2}{c}{$x_3$}\\
                \end{tabular}
            \end{figure}
            Daraus ergibt sich:
            \begin{equation*}
                g_\text{Min}(x) = \textcolor{blue}{x_1} + \textcolor{green}{\overline{x_2} \cdot x_3} + \textcolor{red}{x_2 \cdot \overline{x_3}}
            \end{equation*}
    \end{enumerate}

    \section{Shannon Expansion}
    \setcounter{subsection}{3}
    \subsection{}
    \begin{enumerate}[label=(\alph*)]
        \item
            \begin{eqnarray*}
                f(x_1, x_2, x_3) &=& \overline{x_1} \cdot \overline{x_2} \cdot \overline{x_3} \cdot f(0,0,0) \\
                &&+ \overline{x_1} \cdot \overline{x_2} \cdot x_3 \cdot f(0,0,1) \\
                &&+ \overline{x_1} \cdot x_2 \cdot \overline{x_3} \cdot f(0,1,0) \\
                &&+ \overline{x_1} \cdot x_2 \cdot x_3 \cdot f(0,1,1) \\
                &&+ x_1 \cdot \overline{x_2} \cdot \overline{x_3} \cdot f(1,0,0) \\
                &&+ x_1 \cdot \overline{x_2} \cdot x_3 \cdot f(1,0,1) \\
                &&+ x_1 \cdot x_2 \cdot \overline{x_3} \cdot f(1,1,0) \\
                &&+ x_1 \cdot x_2 \cdot x_3 \cdot f(1,1,1) \\
                &=& \overline{x_1} \cdot \overline{x_2} \cdot \overline{x_3} \cdot 0 \\
                &&+ \overline{x_1} \cdot \overline{x_2} \cdot x_3 \cdot 1 \\
                &&+ \overline{x_1} \cdot x_2 \cdot \overline{x_3} \cdot 1 \\
                &&+ \overline{x_1} \cdot x_2 \cdot x_3 \cdot 1 \\
                &&+ x_1 \cdot \overline{x_2} \cdot \overline{x_3} \cdot 1 \\
                &&+ x_1 \cdot \overline{x_2} \cdot x_3 \cdot 0 \\
                &&+ x_1 \cdot x_2 \cdot \overline{x_3} \cdot 1 \\
                &&+ x_1 \cdot x_2 \cdot x_3 \cdot 0 \\
                &=& \overline{x_1} \cdot \overline{x_2} \cdot x_3 \\
                &&+ \overline{x_1} \cdot x_2 \cdot \overline{x_3} \\
                &&+ \overline{x_1} \cdot x_2 \cdot x_3\\
                &&+ x_1 \cdot \overline{x_2} \cdot \overline{x_3} \\
                &&+ x_1 \cdot x_2 \cdot \overline{x_3} \\
            \end{eqnarray*}
        \item
            \begin{eqnarray*}
                f(x_1, x_2, x_3) &=& \overline{x_1} \cdot \overline{x_2} \cdot \overline{x_3} \cdot f(0,0,0) \\
                &&+ \overline{x_1} \cdot \overline{x_2} \cdot x_3 \cdot f(0,0,1) \\
                &&+ \overline{x_1} \cdot x_2 \cdot \overline{x_3} \cdot f(0,1,0) \\
                &&+ \overline{x_1} \cdot x_2 \cdot x_3 \cdot f(0,1,1) \\
                &&+ x_1 \cdot \overline{x_2} \cdot \overline{x_3} \cdot f(1,0,0) \\
                &&+ x_1 \cdot \overline{x_2} \cdot x_3 \cdot f(1,0,1) \\
                &&+ x_1 \cdot x_2 \cdot \overline{x_3} \cdot f(1,1,0) \\
                &&+ x_1 \cdot x_2 \cdot x_3 \cdot f(1,1,1) \\
                &=& \overline{x_1} \cdot \overline{x_2} \cdot \overline{x_3} \cdot 1 \\
                &&+ \overline{x_1} \cdot \overline{x_2} \cdot x_3 \cdot 1 \\
                &&+ \overline{x_1} \cdot x_2 \cdot \overline{x_3} \cdot 1 \\
                &&+ \overline{x_1} \cdot x_2 \cdot x_3 \cdot 1 \\
                &&+ x_1 \cdot \overline{x_2} \cdot \overline{x_3} \cdot 0 \\
                &&+ x_1 \cdot \overline{x_2} \cdot x_3 \cdot 1 \\
                &&+ x_1 \cdot x_2 \cdot \overline{x_3} \cdot 0 \\
                &&+ x_1 \cdot x_2 \cdot x_3 \cdot 0 \\
                &=& \overline{x_1} \cdot \overline{x_2} \cdot \overline{x_3} \\
                &&+ \overline{x_1} \cdot \overline{x_2} \cdot x_3 \\
                &&+ \overline{x_1} \cdot x_2 \cdot \overline{x_3} \\
                &&+ \overline{x_1} \cdot x_2 \cdot x_3\\
                &&+ x_1 \cdot \overline{x_2} \cdot x_3 \\
            \end{eqnarray*}
    \end{enumerate}
\end{document}
