\documentclass[DIN, pagenumber=false, fontsize=11pt, parskip=half]{scrartcl}

\usepackage{amsmath}
\usepackage{amsfonts}
\usepackage{amssymb}
\usepackage{enumitem}
\usepackage[utf8]{inputenc} % this is needed for umlauts
\usepackage[ngerman]{babel} % this is needed for umlauts
\usepackage[T1]{fontenc} 
\usepackage{commath}
\usepackage{xcolor}
\usepackage{booktabs}
\usepackage{float}
\usepackage{tikz-timing}
\usepackage{tikz}
\usepackage{multirow}

\usetikzlibrary{calc,shapes.multipart,chains,arrows}

\title{Grundlagen der Rechnerarchitektur}
\author{Tim Luchterhand, Paul Nykiel (Abgabegruppe 117)}

\newcommand{\boolshitKgV}[1]{\text{kgV}(#1,30) &=& 30\\}
\newcommand{\boolshitggT}[1]{\text{ggT}(#1,30) &=& #1\\}

\begin{document}
    \maketitle
    \textbf{Hinweis: } Jede Zahl, bei der keine Basis spezifiziert ist, ist im 10er System zu interpretieren, sofern nicht anders angegeben.
    \section{Schaltalgebra}
    \subsection{}
    \begin{enumerate}[label = (\alph*)]
        \item Die Schaltalgebra ist eine Teilmenge der Boolschen Algebra mit einer zweiwertigen Trägermenge. 
            Das heißt es gibt nur zwei möglichen Werte, anstatt beliebig vielen Werten bei einer Schaltalgebra.
        \item 
            \begin{enumerate}[label = (\roman*)]
                \item 
                    Nur mit NAND-Gattern:
                    \begin{equation*}
                        x_1 \cdot x_2 = \overline{\overline{x_1 \cdot x_2}} = \overline{\overline{x_1 \cdot x_2} \cdot \overline{x_1 \cdot x_2}}
                    \end{equation*}
                    Nur mit NOR-Gattern:
                    \begin{equation*}
                        x_1 \cdot x_2 = \overline{\overline{x_1 \cdot x_2}} = \overline{\overline{x_1} + \overline{x_2}}= \overline{\overline{x_1 + 0} + \overline{x_2 + 0}}
                    \end{equation*}
                \item
                    Nur mit NAND-Gattern:
                    \begin{equation*}
                        x_1 \cdot \overline{x_2} + \overline{x_1} \cdot x_2 = x_1 \cdot \overline{x_2 \cdot x_2} + \overline{x_1 \cdot x_1} \cdot x_2  
                        = \overline{\overline{x_1 \cdot \overline{x_2 \cdot x_2} + \overline{x_1 \cdot x_1} \cdot x_2}}
                        = \overline{\overline{x_1 \cdot \overline{x_2 \cdot x_2}} \cdot \overline{ \overline{x_1 \cdot x_1} \cdot x_2}}
                    \end{equation*}
                    Nur mit NOR-Gattern:
            \end{enumerate}
    \end{enumerate}
\end{document}
