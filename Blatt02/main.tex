\documentclass[DIN, pagenumber=false, fontsize=11pt, parskip=half]{scrartcl}

\usepackage{amsmath}
\usepackage{amsfonts}
\usepackage{amssymb}
\usepackage{enumitem}
\usepackage[utf8]{inputenc} % this is needed for umlauts
\usepackage[ngerman]{babel} % this is needed for umlauts
\usepackage[T1]{fontenc} 
\usepackage{commath}
\usepackage{xcolor}
\usepackage{booktabs}
\usepackage{float}
\usepackage{tikz-timing}
\usepackage{tikz}
\usepackage{multirow}

\usetikzlibrary{calc,shapes.multipart,chains,arrows}

\title{Grundlagen der Rechnerarchitektur}
\author{Tim Luchterhand, Paul Nykiel (Abgabegruppe 117)}

\begin{document}
    \maketitle
    \textbf{Hinweis: } jede Zahl, bei der keine Basis angegeben ist, ist im 10er System.
    \section{Zahlensysteme}
    \begin{enumerate}
        \item  $b = 1$ sowie $\Sigma_{b1} = \{|\}$ die Niederwertigste Ziffer steht in der Mitte. Die Null wird durch $\thicksim$ dargestellt.
        \item $15 = ||||| ||||| |||||$
        \item $7 = || |||||$
        \item $101 = | 
            ||||| ||||| 
            ||||| ||||| 
            ||||| ||||| 
            ||||| ||||| 
            ||||| ||||| 
            ||||| ||||| 
            ||||| ||||| 
            ||||| ||||| 
            ||||| ||||| 
            ||||| |||||$
        \item $0 = \thicksim$

    \end{enumerate}
\end{document}
