\documentclass[DIN, pagenumber=false, fontsize=11pt, parskip=half]{scrartcl}

\usepackage{amsmath}
\usepackage{amsfonts}
\usepackage{amssymb}
\usepackage{enumitem}
\usepackage[utf8]{inputenc} % this is needed for umlauts
\usepackage[ngerman]{babel} % this is needed for umlauts
\usepackage[T1]{fontenc} 
\usepackage{commath}
\usepackage{xcolor}
\usepackage{booktabs}
\usepackage{float}
\usepackage{tikz-timing}
\usepackage{tikz}
\usepackage{multirow}

\usetikzlibrary{calc,shapes.multipart,chains,arrows}

\title{Grundlagen der Rechnerarchitektur}
\author{Tim Luchterhand, Paul Nykiel (Abgabegruppe 117)}

\newcommand{\bullshit}[2]{#1-er System:
            \begin{equation*}
                {1001100}_{#1} = #1^6 \cdot 1 + #1^5 \cdot 0 + #1^4 \cdot 0 + #1^3 \cdot 1 + #1^2 \cdot 1 +  #1^1 \cdot 0 + #1^0 \cdot  0 = #2
            \end{equation*}
            }


\begin{document}
    \maketitle
    \textbf{Hinweis: } Jede Zahl, bei der keine Basis spezifiziert ist, ist im 10er System zu interpretieren, sofern nicht anders angegeben.
    \section{Zahlensysteme}
    \subsection{}
    \begin{enumerate}[label = (\alph*)]
        \item  $b = 1$ sowie $\Sigma_{b1} = \{|\}$ die Niederwertigste Ziffer steht in der Mitte. Die Null wird durch $\thicksim$ dargestellt.
        \item $15 = {||||| ||||| |||||}_1$
        \item $7 = {|| |||||}_1$
        \item $101 = {| 
            ||||| ||||| 
            ||||| ||||| 
            ||||| ||||| 
            ||||| ||||| 
            ||||| ||||| 
            ||||| ||||| 
            ||||| ||||| 
            ||||| ||||| 
            ||||| ||||| 
            ||||| |||||}_1$
        \item $0 = {\thicksim}_1$
    \end{enumerate}
    \section{Arithmetik natürlicher und ganzer Zahlen}
    \setcounter{subsection}{1}
    \subsection{}
    \begin{enumerate}[label = (\alph*)]
        \item 
            \begin{align*}
                & 1037 \\
                + & \hspace{0.0cm} 3802  \\
                %& \hspace{0.2cm} {\scriptstyle 1\, 1} \\
                \cline{1-2}  
                = &4839  
            \end{align*}
        \item 
            \begin{align*}
                & 9548\\
                - & \hspace{0.0cm} 3027  \\
                %& \hspace{0.2cm} {\scriptstyle 1\, 1} \\
                \cline{1-2}  
                = &6521  
            \end{align*}
        \item Folgende Rechnung ist im Dualsystem:
            \begin{align*}
                & \hspace{0.1cm} 1010110\\
                + & \hspace{0.1cm}1010001 \\
                & \hspace{0.0cm} {\scriptstyle 1\, \, \, 1} \\
                \cline{1-2}  
                = &10101111
            \end{align*}
        \item Folgende Rechnung ist im Dualsystem:
            \begin{align*}
                & \hspace{0.0cm} 1010111\\
                + & \hspace{0.0cm}1000101 \\
                \cline{1-2}  
                = &0010010
            \end{align*}
    \end{enumerate}
    \subsection{}
    \begin{enumerate}[label = (\alph*)]
        \item \bullshit{3}{765}
        \item \bullshit{4}{4176}
        \item \bullshit{5}{15775}
        \item \bullshit{6}{46908}
        \item \bullshit{7}{118041}
        \item \bullshit{8}{262720}
        \item \bullshit{9}{532251}
        \item \bullshit{10}{1001100}
        \item \bullshit{11}{1773013}
        \item \bullshit{12}{2987856}
    \end{enumerate}
    \subsection{}
    In folgender Reihenfolge: Zahl, Vorzeichenbehaftete Binärzahl, Einerkomplement, Zweierkomplement. Alle Zahlen werden als 16-bit Breit angenommen.
    \begin{align*}
        -861 &=& -0000\ 0011\ 0101\ 1101 &=& 1111\ 1100\ 1010\ 0010 &=& 1111\ 1100\ 1010\ 0011\\
        {765}_8 &=& 0000\ 0001\ 1111\ 0101 &=& 0000\ 0001\ 1111\ 0101 &=& 0000\ 0001\ 1111\ 0101 \\ 
        -{210}_3 &=&- 0000\ 0000\ 0001\ 0101 &=& 1111\ 1111\ 1110\ 1010 &=& 1111\ 1111\ 1110\ 1011
    \end{align*}
    
    \subsection{}
    \begin{enumerate}[label = (\alph*)]
        \item Nebenrechnung: 
            \begin{equation*}
                {765}_{18} = {0000\ 1001\ 0100\ 1101}_2
            \end{equation*}
            Vorzeichenbehaftete Binärzahl:
            \begin{align*}
                & \hspace{0.0cm} 0000\ 1001\ 0100\ 1101 \\
                - & \hspace{0.0cm}0000\ 0000\ 0001\ 0101\\
                & \hspace{2.2cm} {\scriptstyle 1} \\
                \cline{1-2}  
                = & 0000\ 1001\ 0011\ 1000
            \end{align*}
            Zweierkomplement (binär):
            \begin{align*}
                & \hspace{0.2cm} 0000\ 1001\ 0100\ 1101 \\
                + & \hspace{0.2cm}1111\ 1111\ 1110\ 1011\\
                & \hspace{0.2cm} {\scriptstyle1\, 1\, 1\, 1\, \, \, 1\, 1\, 1\, 1\, \, \, 1\, \, \, \, \, \, \, \, \, 1\, \, \, 1\, 1\, 1} \\
                \cline{1-2}  
                = & 1\ 0000\ 1001\ 0011\ 1000
            \end{align*}
            Einerkomplement (binär, zuerst nur addieren):
            \begin{align*}
                & \hspace{0.2cm} 0000\ 1001\ 0100\ 1101 \\
                + & \hspace{0.2cm}1111\ 1111\ 1110\ 1010\\
                & \hspace{0.2cm} {\scriptstyle1\, 1\, 1\, 1\, \, \, 1\, 1\, 1\, 1\, \, \, 1\, \, \, \, \, \, \, \, \, 1} \\
                \cline{1-2}  
                = & 1\ 0000\ 1001\ 0011\ 0111
            \end{align*}
            Ergebniss der Subtraktion, durch addieren von 1:
            \begin{align*}
                & 1\ 0000\ 1001\ 0011\ 0111\\
                + & \hspace{0.3cm}0000\ 0000\ 0000\ 0001\\
                & \hspace{3.03cm} {\scriptstyle1\, 1\, 1} \\
                \cline{1-2}  
                = & 1\ 0000\ 1001\ 0011\ 1000
            \end{align*}
        \item Vorzeichenbehaftete Binärzahl :
            \begin{align*}
                - & \hspace{0.0cm}0000\ 0011\ 0101\ 1101\\
                - & \hspace{0.0cm}0000\ 0000\ 0001\ 0101\\
                %& \hspace{2.2cm} {\scriptstyle 1} \\
                \cline{1-2}  
                = -& 0000\ 0011\ 0100\ 1000
            \end{align*}
            Zweierkomplement (binär):
            \begin{align*}
                & \hspace{0.2cm} 1111\ 1100\ 1010\ 0011\\
                + & \hspace{0.2cm}1111\ 1111\ 1110\ 1011\\
                & \hspace{0.0cm} {\scriptstyle 1\, \, 1\, 1\, 1\, 1\, \, 1\, 1\, 1\, 1\, \, \, 1\, 1\hspace{0.75cm}1\, 1} \\
                \cline{1-2}  
                = & 1\ 1111\ 1100\ 1000\ 1110
            \end{align*}
            Einerkomplement (binär, zuerst nur addieren):
            \begin{align*}
                & \hspace{0.2cm} 1111\ 1100\ 1010\ 0010\\
                + & \hspace{0.2cm}1111\ 1111\ 1110\ 1010\\
                & \hspace{0.0cm} {\scriptstyle 1\, \, 1\, 1\, 1\, 1\, \, 1\, 1\, 1\, 1\, \, \, 1\, 1\hspace{0.75cm}1} \\
                \cline{1-2}  
                = & 1\ 1111\ 1100\ 1000\ 1100
            \end{align*}
            Ergebniss der Subtraktion, durch addieren von 2 (zwei Zahlen im Einerkomplement):
            \begin{align*}
                & 1\ 1111\ 1100\ 1000\ 1100 \\
                + & \hspace{0.3cm}0000\ 0000\ 0000\ 0010\\
                %& \hspace{3.03cm} {\scriptstyle1\, 1\, 1} \\
                \cline{1-2}  
                = & 1\ 1111\ 1100\ 1000\ 1110
            \end{align*}
    \end{enumerate}

    \subsection{}
    \begin{enumerate}[label = (\alph*)]
        \item Das Binary-Code-Decimal Format codiert die einzelnen Stellen der Zahl im Dezimalsystem, in dem die einzelnen Ziffern binär dargestellt werden.
            Das heißt jede Ziffer der Zahl im Zehnersystem  wird mit 4 bit repräsentiert.
        \item 
            \begin{equation*}
                253 = 2 \cdot 10^2 + 5 \cdot 10^1 + 3 \cdot 10^0 = {0010}_2 \cdot 10^2 + {0101}_2 \cdot 10^1 + {0011}_2 \cdot 10^0 = {0010\ 0101\ 0011}_\text{BCD}
            \end{equation*}
        \item 
            \begin{equation*}
                {0011\ 0101\ 0111}_\text{BCD} = 3 \cdot 10^2 + 5 \cdot 10^1 + 7 \cdot 10^0 = 357
            \end{equation*}
    \end{enumerate}
\end{document}
