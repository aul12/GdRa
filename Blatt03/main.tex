\documentclass[DIN, pagenumber=false, fontsize=11pt, parskip=half]{scrartcl}

\usepackage{amsmath}
\usepackage{amsfonts}
\usepackage{amssymb}
\usepackage{enumitem}
\usepackage[utf8]{inputenc} % this is needed for umlauts
\usepackage[ngerman]{babel} % this is needed for umlauts
\usepackage[T1]{fontenc} 
\usepackage{commath}
\usepackage{xcolor}
\usepackage{booktabs}
\usepackage{float}
\usepackage{tikz-timing}
\usepackage{tikz}
\usepackage{multirow}

\usetikzlibrary{calc,shapes.multipart,chains,arrows}

\title{Grundlagen der Rechnerarchitektur}
\author{Tim Luchterhand, Paul Nykiel (Abgabegruppe 117)}

\begin{document}
    \maketitle
    \textbf{Hinweis: } Jede Zahl, bei der keine Basis spezifiziert ist, ist im 10er System zu interpretieren, sofern nicht anders angegeben.
    \section{Zahlendarstellung}
    \subsection{}
    \subsubsection{Gleitkommazahlen}
    Gleitkommazahlen sind flexibel im Bezug auf die Größenordnung der Zahl, das heißt man kann sehr große und sehr kleine Zahlen gut darstellen und 
    ohne zu wissen in welcher Größenordnung sich die Zahlen befinden.

    Mathematische Operationen auf Gleitkommazahlen sind vergleichsweise aufwendig und erfordern häufig ein spezielles Rechenwerk.
    \subsubsection{Festkommazahlen}
    Mathematische Operationen auf Festkommazahlen lassen sich bereits mit einem einfachen Ganzzahlrechenwerk durchführen.

    Bei Festkommazahlen muss im Vorraus bekannt sein, in welcher Größenordnung sich die Zahl befinden wird, da jeweils eine feste Anzahl Bits für 
    Vor- und Nachkommastellen reserviert sind.

    \subsection{}
    \begin{equation*}
        \frac{43}{30} = 1 + \frac{13}{30} = 1 + \frac{1}{2+\frac{4}{30}} = \frac{1}{2+\frac{1}{7+\frac{2}{4}}} = \frac{1}{2+\frac{1}{7+\frac{1}{2}}}    
    \end{equation*}
    \begin{equation*}
        \frac{55}{19} = 2 + \frac{17}{19} = 1 + \frac{1}{1 + \frac{2}{17}} = \frac{1}{1 +  \frac{1}{8 + \frac{1}{2}}}
    \end{equation*}

    \section{Gleitkommaarithmetik}
    \setcounter{subsection}{2}
    \subsection{}
    \begin{enumerate}[label=(\alph*)]
        \item
            Größte Zahl: Mantisse Maximal (${22222}_3 = 242$), Exponent Maximal (${222}_3 - 13 = 13$) und Vorzeichen positiv.
            \begin{equation*}
                z_\text{max} = + 242 \cdot 3^{13} = 385826166 
            \end{equation*}
            Kleinste Zahl: wie größte Zahl nur mit negativem Vorzeichen
            \begin{equation*}
                z_\text{min} = - z_\text{max} = -385826166
            \end{equation*}
        \item 
            \begin{enumerate}[label=\roman*.]
                \item 
                    \begin{equation*}
                        2.25 = 1.125 \cdot 2^1 = (-1)^0 \cdot 2^{(4-3)} \cdot (1 + 2^{-3}) = {0\ 100\ 00100}_\text{Gleitkomma} 
                    \end{equation*}
                    \begin{equation*}
                        5.625 = 1.40625 \cdot 2^2 = (-1)^0 \cdot 2^{(5-3)} \cdot (1 + 2^{-2} + 2^{-3} + 2^{-5}) = {0\ 101\ 01101}_\text{Gleitkomma} 
                    \end{equation*}
                    Folgende Rechnung im Gleitkommaformat:
                    \begin{equation*}
                        0\ 100\ 00100 + 0\ 101\ 01101 \stackrel{\text{Exponent angleichen}}{=} 0\ 101\ 10010 + 0\ 100\ 01101 = 0\ 101\ 01111
                    \end{equation*}
                    Da die Mantisse nur 5 Bit lang ist kann die Zahl nicht ohne Fehler dargestellt werden.
                \item 
                    \begin{equation*}
                        2.5 = 1.25 \cdot 2^1 = (-1)^0 \cdot 2^{(4-3)} \cdot (1 + 2^{-2}) = {0\ 100\ 01000}_\text{Gleitkomma}
                    \end{equation*}
                    \begin{equation*}
                        16.8 = 1.05 \cdot 2^4 \approx (-1)^0 \cdot 2^{(7-3)} \cdot (1 + 2^{-5} + 2^{-6} + 2^{-9} + \ldots) 
                        \approx {0\ 111\ 00001}_\text{Gleitkomma}
                    \end{equation*}
                    Fünf Bit Mantisse reichen hier nicht aus, um alle Nachkommastellen darzustellen.

                    Folgende Rechnung im Gleitkommaformat:
                    \begin{equation*}
                        0\ 100\ 00100 \cdot 0\ 111\ 00001 =
                        0\ 000\ 10101
                    \end{equation*}
                    Die Exponenten werden addiert und dann wieder normiert (Bias abziehen). Dabei kommt es zu einem Overflow, 
                    der Exponent ist eigentlich ${1000}_2$, es werden aber nur die unteren drei Bits gespeichert. Bei IEEE-754 gibt es für diesen Fall den
                    Spezialwert infinity.
                \item 
                    \begin{equation*}
                        -6.375 = -1 \cdot 2^2 \cdot 1.59375 = (-1)^1 \cdot 2^{(5-3)} \cdot (1 + 2^{-1} + 2^{-4} + 2^{-5}) 
                        = {1\ 101\ 10011}_\text{Gleitkomma}
                    \end{equation*}
                    \begin{equation*}
                        4.125 = 1.03125 \cdot 2^2 = (-1)^0 \cdot 2^{(5-3)} \cdot (1 + 2^{-5}) = {0\ 101\ 00001}_\text{Gleitkomma}
                    \end{equation*}
                    
                    Folgende Rechnung im Gleitkommaformat:
                    \begin{equation*}
                        {1\ 101\ 10011} + {0\ 101\ 00001} = {1\ 100\ 00100}
                    \end{equation*}
            \end{enumerate}
    \end{enumerate}

    \section{Festkommarithmetik}
    \setcounter{subsection}{3}
    \subsection{}
    \begin{enumerate}[label=(\alph*)]
        \item 
            \begin{enumerate}[label=(\roman*)]
                \item 
                    \begin{equation*}
                        8.875 = 2^3 + 2^{-1} + 2^{-2} + 2^{-3} = {01000,11100}_2
                    \end{equation*}
                \item
                    \begin{equation*}
                        31.125 = 2^4 + 2^3 + 2^2 + 2^1  + 2^0 + 2^{-3} = {11111,00100}_2
                    \end{equation*}
            \end{enumerate}
        \item 
            \begin{enumerate}[label=(\roman*)]
                \item
                    \begin{equation*}
                        -5.625 = -(2^2 + 2^0 + 2^{-1} + 2^{-3}) = {-00101,10100}_2
                    \end{equation*}
                    \begin{equation*}
                        7.65625 = 2^2 + 2^1 + 2^0 +2^{-1} + 2^{-3} + 2^{-5} = {00111,10101}_2
                    \end{equation*}

                    Folgende Rechnung im Festkommaformat:
                    \begin{equation*}
                        -00101,10100 + 00111,10101 = 10,00001
                    \end{equation*}
                \item
                    \begin{equation*}
                        7.03125 = 2^2 + 2^1 + 2^0 + 2^{-5} = {00111,00001}_2
                    \end{equation*}
                    \begin{equation*}
                        -(2.03125) = -(2^1 + 2^{-5}) = -{00010,00001}_2
                    \end{equation*}
                    
                    Folgende Rechnung im Festkommaformat:
                    \begin{equation*}
                        00111,00001 - (-00010,00001) = 00111,00001 + 00010,00001 = 01001,00010
                    \end{equation*}

                \item 
                    \begin{equation*}
                        -5.15625 = -(2^2 + 2^1 + 2^{-3} + 2^{-5}) = -{00110,00101}_2
                    \end{equation*}
                    \begin{equation*}
                        0.25 = 2^{-2} = {00000,01000}_2
                    \end{equation*}

                    Folgende Rechnung im Festkommaformat:
                    \begin{equation*}
                        -{00110,00101}_2 : {00000,01000}_2 = -{11000,10100}_2
                    \end{equation*}

                \item
                    \begin{equation*}
                        4.0 = 2^2 = {00100,00000}_2
                    \end{equation*}
                    \begin{equation*}
                        -2.0625 = -(2^1 + 2^{-4}) = -{00010,00010}_2
                    \end{equation*}

                    Folgende Rechung im Festkommaformat:
                    \begin{equation*}
                        {00100,00000}_2 \cdot -{00010,00010}_2 = -{01000,01000}
                    \end{equation*}
            \end{enumerate}
    \end{enumerate}
\end{document}
