\documentclass[DIN, pagenumber=false, fontsize=11pt, parskip=half]{scrartcl}

\usepackage{amsmath}
\usepackage{amsfonts}
\usepackage{amssymb}
\usepackage{enumitem}
\usepackage[utf8]{inputenc} % this is needed for umlauts
\usepackage[ngerman]{babel} % this is needed for umlauts
\usepackage[T1]{fontenc} 
\usepackage{commath}
\usepackage{xcolor}
\usepackage{booktabs}
\usepackage{float}
\usepackage{tikz-timing}
\usepackage{tikz}
\usepackage{multirow}

\usetikzlibrary{calc,shapes.multipart,chains,arrows}

\title{Grundlagen der Rechnerarchitektur}
\author{Paul Nykiel}

\begin{document}
    \maketitle
    \section{Einleitungsfragen}
    \begin{enumerate}[label=(\alph*)]
            \item Unter Rechnerarchitektur stelle ich mir den Aufbau und die Funktionsweise eines Computers vor. Vor allem die digitalen (Halbleiter-) Schaltungen die Realisiert werden müssen, um Logik-Gatter zu bauen, und daraus dann Rechenwerke und Speicher zu konstruieren.
            \item Für mich sind die Grundlagen der Rechnerarchitektur die theoretischen Grundlagen zu Digitalen-Schaltungen um solche zu optimieren, sowie die technischen Grundlagen zu Halbleitern sowie zur Herstellung von diesen.
            \item Ich erhoffe mir von der Vorlesung mehr über den Aufbau von Computern zu erfahren, sowie wie sich Computer auf niedriger Ebene mit Assembler programmieren lassen.
    \end{enumerate}

    \section{Historische Entwicklung}
    \begin{enumerate}[label=(\alph*)]
        \item Durch die Reibung der mechanischen Bauteile wird auch schon für einfache Rechner sehr viel Kraft benötigt und folglich viel Energie verbraucht.
            Komplexere Rechner konnten so folglich nicht realisiert werden. Außerdem waren die mechanischen Rechner äußerst Fehleranfällig, da sich die 
            Schaltglieder verhaken konnten.
        \item Vor der Elektronischen Speicherung wurden Daten auf Lochkarten gespeichert. Diese wurden 1880 von Hermann Hollerith nach dem Vorbild von
            gelochten Fahrkarten entwickelt. Durch das Lochmuster wurden die verschiedenen Daten codiert.
        \item Die neuen Rechner waren mit den alten Rechnern nicht binärkompatibel, das heißt die Rechner unterstützen nicht die gleichen Befehle.
            Zudem hat oftmals die Wortbreite zwischen den Architekturen variiert.
        \item Die (elektro-) mechanischen Rechner wurden durch Rechner auf Basis von Halbleitern ersetzt. Diese wurden zuerst mit Germanium-Dioden und später mit
            Transistoren gebaut.
        \item Da das kaufmännische Rechnen vor allem viele Operationen mit ganzen Zahlen (Cents)  erforderten waren die kaufmännischen Rechner darauf optimiert.
            Für wissenschaftliches Rechnen sind oftmals Gleitkommaoperationen notwendig, diese liesen sich auf kaufmännischen Rechner nur langsam durchführen.
    \end{enumerate}
\end{document}
