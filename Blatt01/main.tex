\documentclass[DIN, pagenumber=false, fontsize=11pt, parskip=half]{scrartcl}

\usepackage{amsmath}
\usepackage{amsfonts}
\usepackage{amssymb}
\usepackage{enumitem}
\usepackage[utf8]{inputenc} % this is needed for umlauts
\usepackage[ngerman]{babel} % this is needed for umlauts
\usepackage[T1]{fontenc} 
\usepackage{commath}
\usepackage{xcolor}
\usepackage{booktabs}
\usepackage{float}
\usepackage{tikz-timing}
\usepackage{tikz}
\usepackage{multirow}

\usetikzlibrary{calc,shapes.multipart,chains,arrows}

\title{Grundlagen der Rechnerarchitektur}
\author{Tim Luchterhand, Paul Nykiel (Abgabegruppe 117)}

\begin{document}
    \maketitle
    \textbf{Hinweis: } jede Zahl, bei der keine Basis angegeben ist, ist im 10er System zu interpretieren.
    \section{Zahlensysteme}
    \begin{enumerate}
        \item In additiven Zahlensysteme ist Multiplikation und Division nur schwer durchführbar.
        \item 
            \begin{enumerate}
                \item $7 \cdot 1 + 20 \cdot 2 + 360 \cdot 6 = 2207$
                \item $2 \cdot 1 + 20 \cdot 8 + 360 \cdot 4 + 7200 \cdot 5 = 37602$
            \end{enumerate}
        \item 
            \begin{enumerate}
                \item $\text{LI} = 50 + 1 = 51 = 32 + 16 + 2 + 1 = {110011}_2$
                \item $\text{CDXIX} = 500 - 100 + 10 + 10 - 1 = 419 = 256 + 128 + 32 + 2 + 1 = {110100011}_2$
            \end{enumerate}
        \item 
            \begin{enumerate}
                \item 
                    \begin{eqnarray*}
                        849 = 512 + 256 + 64 + 16 + 1 &=& {1101010001}_2\\
                        {11\ 0101\ 0001}_2 = {3\ 5\ 1}_{16} &=& {351}_{16}\\
                        {1\ 101\ 010\ 001}_2 = {1\ 5\ 2\ 1}_{8} &=& {1521}_8
                    \end{eqnarray*}
                \item 
                    \begin{eqnarray*}
                        {1022}_{3} = {1 \cdot 3^3 + 0 \cdot 3^2 + 2 \cdot 3^1 + 2 \cdot 3^0}_{10} = 27 + 6 + 2 &=& 35\\
                        35 = {32 + 2 + 1}_{10} &=& {100011}_{2}\\
                        {10\ 0011}_2 = {2\ 3}_{16} &=& {23}_{16} \\
                        {100\ 011}_2 = {4\ 3}_{8} &=& {43}_{8}
                    \end{eqnarray*}
            \end{enumerate}
        \item 
            \begin{enumerate}
                \item 
                    \begin{eqnarray*}
                        {1001101}_2 &=& 1 \cdot 2^6 + 0 \cdot 2^5 + 0 \cdot 2^4 + 1 \cdot 2^3 + 1 \cdot 2^2 + 0 \cdot 2^1 + 1 \cdot 2^0\\ 
                        &=& 64 + 8 + 4 + 1= 77
                    \end{eqnarray*}
                \item 
                    \begin{eqnarray*}
                        {100101101}_3 &=& 1 \cdot 3^8 + 0 \cdot 3^7 + 0 \cdot 3^6 + 1 \cdot 3^5 + 0 \cdot 3^4 + 1 \cdot 3^3 + 1 \cdot 3^2 + 0 \cdot 3^1 + 1 \cdot 3^0\\  
                        &=& 6561 + 243 + 27 + 9 + 1 = 6841
                    \end{eqnarray*}
                \item 
                    \begin{equation*}
                        {461}_{8} = 4 \cdot 8^2 + 6 \cdot 8^1 + 1 \cdot 8^0 = 265 + 48 + 1 = 305
                    \end{equation*}
                \item 
                    \begin{equation*}
                        {1AF4}_{16} = 1 \cdot 16^3 + 10 \cdot 16^2 + 15 \cdot 16^1 + 4 \cdot 16^0 = 6900
                    \end{equation*}
            \end{enumerate}
        \item 
            \begin{enumerate}
                \item 
                    \begin{equation*}
                        {AFFE}_{16} = {1010\ 1111\ 1111\ 1110}_2 = {1\ 010\ 111\ 111\ 111\ 110}_2 = {1\ 2\ 7\ 7\ 7\ 6}_8 = {12776}_8
                    \end{equation*}
                \item
                    \begin{equation*}
                        {F903}_{16} = {1111\ 1001\ 0000\ 0011}_2 = {1\ 111\ 100\ 100\ 000\ 011}_2 = {1\ 7\ 4\ 4\ 0\ 3}_8 = {174403}_8
                    \end{equation*}
                \item
                    \begin{equation*}
                        {3F8B}_{16} = {0011\ 1111\ 1000\ 1011}_2 = {0\ 011\ 111\ 110\ 001\ 011}_2 = {0\ 3\ 7\ 6\ 1\ 3}_8 = {37613}_8
                    \end{equation*}
            \end{enumerate}
        \item 
            \begin{enumerate}
                \item 
                    \begin{eqnarray*}
                        {753}_8 = {111\ 101\ 011}_2 &=& {1\ 1110\ 1011}_2 \\
                        {753}_8 = 7 \cdot 8^2 + 5 \cdot 8^1 + 3 \cdot 8^0 = 448 + 40 + 3 &=& 491\\
                        121 \cdot 4 + 11 \cdot 0 + 1 \cdot 7 &=& {407}_{11}
                    \end{eqnarray*}
                \item 
                    \begin{eqnarray*}
                        {372}_8 = {011\ 111\ 010}_2 &=& {1111\ 1010}_2 \\
                        {1111\ 1010}_2 = 2^8 - 2^4 - 2^0 &=& 250 \\
                        250 = 2 \cdot 11^2 + 0 \cdot 11^1 + 8 \cdot 11^0 &=& {208}_{11}
                    \end{eqnarray*}
                \item 
                    \begin{eqnarray*}
                        {3516}_8 = {011\ 101\ 001\ 110}_2 &=& {0111\ 0100\ 1110}_2 \\
                        {3516}_8 = 3 \cdot 8^3 + 5 \cdot 8^2 + 1 \cdot 8^1 + 6 \cdot 8^0 &=& 1870 \\
                        1870 = 1 \cdot 11^3 + 4 \cdot 11^2 + 5 \cdot 11^1 + 0 \cdot 11^0 &=& {1450}_{11}
                    \end{eqnarray*}
            \end{enumerate}
        \item Annahme mit $\mathcal{Z}$ wird $\mathbb{Z}$ gemeint, also die Menge der ganzen Zahlen. Dann gilt (nach Axiom 2):
            \begin{equation*}
                (-1) \in \mathbb{Z} \Rightarrow  (-1)' \in \mathbb{Z} \Rightarrow 0 \in \mathbb{Z}
            \end{equation*}
            Dies stellt einen Widerspruch zu Axiom 3 dar.
    \end{enumerate}
\end{document}
