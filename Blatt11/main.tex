\documentclass[DIN, pagenumber=false, fontsize=11pt, parskip=half]{scrartcl}

\usepackage{amsmath}
\usepackage{amsfonts}
\usepackage{amssymb}
\usepackage{enumitem}
\usepackage[utf8]{inputenc} % this is needed for umlauts
\usepackage[ngerman]{babel} % this is needed for umlauts
\usepackage[T1]{fontenc} 
\usepackage{commath}
\usepackage{xcolor}
\usepackage{booktabs}
\usepackage{float}
\usepackage{tikz-timing}
\usepackage{tikz}
\usepackage{multirow}
\usepackage{colortbl}
\usepackage{xstring}
\usepackage{circuitikz}
\usepackage{listings} % needed for the inclusion of source code
\usepackage{mips}


\usetikzlibrary{calc,shapes.multipart,chains,arrows}

\definecolor{dkgreen}{rgb}{0,0.6,0}
\definecolor{gray}{rgb}{0.5,0.5,0.5}
\definecolor{mauve}{rgb}{0.58,0,0.82}

\lstset{ %
  language=[mips]Assembler,       % the language of the code
  basicstyle=\footnotesize,       % the size of the fonts that are used for the code
  numbers=left,                   % where to put the line-numbers
  numberstyle=\tiny\color{gray},  % the style that is used for the line-numbers
  stepnumber=1,                   % the step between two line-numbers. If it's 1, each line
                                  % will be numbered
  numbersep=5pt,                  % how far the line-numbers are from the code
  backgroundcolor=\color{white},  % choose the background color. You must add \usepackage{color}
  showspaces=false,               % show spaces adding particular underscores
  showstringspaces=false,         % underline spaces within strings
  showtabs=false,                 % show tabs within strings adding particular underscores
  frame=single,                   % adds a frame around the code
  rulecolor=\color{black},        % if not set, the frame-color may be changed on line-breaks within not-black text (e.g. commens (green here))
  tabsize=4,                      % sets default tabsize to 2 spaces
  captionpos=b,                   % sets the caption-position to bottom
  breaklines=true,                % sets automatic line breaking
  breakatwhitespace=false,        % sets if automatic breaks should only happen at whitespace
  title=\lstname,                 % show the filename of files included with \lstinputlisting;
                                  % also try caption instead of title
  keywordstyle=\color{blue},          % keyword style
  commentstyle=\color{dkgreen},       % comment style
  stringstyle=\color{mauve},         % string literal style
  escapeinside={\%*}{*)},            % if you want to add a comment within your code
  morekeywords={*,...}               % if you want to add more keywords to the set
}

\title{Grundlagen der Rechnerarchitektur}
\author{Tim Luchterhand, Paul Nykiel (Abgabegruppe 117)}

\begin{document}
    \maketitle
    \section{}
    \begin{enumerate}[label=(\alph*)]
        \item $ $
            \begin{figure}[H]
                \centering
                \begin{tabular}{lp{5cm}p{4cm}}
                    \toprule
                    Befehlsformat & Aufbau & Verwendung \\
                    \midrule
                    Register/ALU-Operationen & ALU-Funktion (Bit 0-5), Verschiebung (Bit 6-10), Zielregister (Bit 11-15), Input Register 1 (Bit 16-20), Input Register 2 (Bit 21-25), OpCode (Bit 26-31) & 
                    Binäre Operationen mit zwei nicht konstanten Register-Operanden (z.B. add, sll) \\
                    Immediate Operationen & Direktoperand (Bit 0-15), Zielregister (Bit 16-20), Indexregister (Bit 21-25), OpCode (Bit 26-31) & Operationen mit Konstanten oder Adressen (z.B. lw, addiu)\\
                    Jump Befehle & Sprungziel (Bit 0-25), Opcode (Bit 26-31) & Sprungbefehle auf Basis von Adressen \\
                    \bottomrule
                \end{tabular}
            \end{figure}
        \item 
            addiu \$t2, \$zero, -11 $\hat{=}$ 9 10 0 -11 $\hat{=}$ 001001 01010 00000 11111111 11110101

            addi \$t1, \$t2, 53 $\hat{=}$ 8 9 10 53 $\hat{=}$ 001000 01001 01010 00000000 00110101

            add \$s0, \$t1, \$zero $\hat{=}$ 0 0 9 16 0 20 $\hat{=}$ 000000 00000 01001 10000 00000 010100

        \item
            000000 01011 10100 01010 00000 100110 $\hat{=}$ 0 11 20 10 0 38 $\hat{=}$ xor \$t2, \$s4, \$t3

            000000 10010 10111 01001 00000 100010 $\hat{=}$ 0 18 23 10 0 34 $\hat{=}$ sub \$t2, \$s7, \$s2

            001000 01001 10000 0001100001100101 $\hat{=}$ 8 9 16 6245 $\hat{=}$ addi \$t1, \$s0, 6245

            000000 01001 01010 10000 00000 100000 $\hat{=}$ 0 9 10 16 0 32 $\hat{=}$ add \$t1, \$t2, \$s0
        \item
            Die Befehle unterscheiden sich im Format (Register und Immediate). Add wird genutzt, um zwei Register aufeinander zu addieren, wohingegen addi ein Register und eine Konstante addiert.

        \item
            \glqq{}The Answer to the Ultimate Question of Life, the Universe, and Everything is 42\grqq{}
    \end{enumerate}

    \section{}
    \begin{enumerate}[label=(\alph*)]
        \item $ $ 
            \lstinputlisting{Aufgabe2a.asm}
        \item $ $ 
            \lstinputlisting{Aufgabe2b.asm}
    \end{enumerate}
\end{document}
