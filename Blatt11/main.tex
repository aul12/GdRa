\documentclass[DIN, pagenumber=false, fontsize=11pt, parskip=half]{scrartcl}

\usepackage{amsmath}
\usepackage{amsfonts}
\usepackage{amssymb}
\usepackage{enumitem}
\usepackage[utf8]{inputenc} % this is needed for umlauts
\usepackage[ngerman]{babel} % this is needed for umlauts
\usepackage[T1]{fontenc} 
\usepackage{commath}
\usepackage{xcolor}
\usepackage{booktabs}
\usepackage{float}
\usepackage{tikz-timing}
\usepackage{tikz}
\usepackage{multirow}
\usepackage{colortbl}
\usepackage{xstring}
\usepackage{circuitikz}

\usetikzlibrary{calc,shapes.multipart,chains,arrows}

\title{Grundlagen der Rechnerarchitektur}
\author{Tim Luchterhand, Paul Nykiel (Abgabegruppe 117)}

\begin{document}
    \maketitle
    \section{}
    \begin{enumerate}[label=(\alph*)]
        \item $ $
            \begin{figure}[H]
                \centering
                \begin{tabular}{lp{7cm}l}
                    \toprule
                    Befehlsformat & Aufbau & Verwendung \\
                    \midrule
                    Register/ALU-Operationen & ALU-Funktion (0-5), Verschiebung (6-10), Zielregister (11-15), Input Register 1 (16-20), Input Register 2 (21-25), OpCode (26-31) & \\
                    Immediate Operationen & Direktoperand (0-15), Zielregister (16-20), Indexregister (21-25), OpCode (26-31) & \\
                    Jump Befehle & Sprungziel (Bit 0-25), Opcode (Bit 26-31) & \\
                    \bottomrule
                \end{tabular}
            \end{figure}
        \item 
            addiu \$t2, \$zero, -11 $\stackrel{\hat{}}{=}$ 9 10 0 -11 $\stackrel{\hat{}}{=}$ 001001 01010 00000 11111111 11110101
    \end{enumerate}
\end{document}
