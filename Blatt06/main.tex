\documentclass[DIN, pagenumber=false, fontsize=11pt, parskip=half]{scrartcl}

\usepackage{amsmath}
\usepackage{amsfonts}
\usepackage{amssymb}
\usepackage{enumitem}
\usepackage[utf8]{inputenc} % this is needed for umlauts
\usepackage[ngerman]{babel} % this is needed for umlauts
\usepackage[T1]{fontenc} 
\usepackage{commath}
\usepackage{xcolor}
\usepackage{booktabs}
\usepackage{float}
\usepackage{tikz-timing}
\usepackage{tikz}
\usepackage{multirow}
\usepackage{colortbl}
\usepackage{xstring}

\usetikzlibrary{calc,shapes.multipart,chains,arrows}

\title{Grundlagen der Rechnerarchitektur}
\author{Tim Luchterhand, Paul Nykiel (Abgabegruppe 117)}

\newcommand{\minTerm}[4]{
    \IfEqCase{#1}{
        {0}{\overline{x_0}\ }
        {1}{x_0}
    }
    \IfEqCase{#2}{
        {0}{\overline{x_1}\ }
        {1}{x_1}
    }
    \IfEqCase{#3}{
        {0}{\overline{x_2}\ }
        {1}{x_2}
    }
    \IfEqCase{#4}{
        {0}{\overline{x_3}\ }
        {1}{x_3}
    }[\PackageError{minTerm}{Values must either be 0 or 1}]
}

\newcommand{\maxTerm}[4]{
    (
    \IfEqCase{#1}{
        {0}{\overline{x_0} + }
        {1}{x_0 + }
    }
    \IfEqCase{#2}{
        {0}{\overline{x_1} + }
        {1}{x_1 + }
    }
    \IfEqCase{#3}{
        {0}{\overline{x_2} + }
        {1}{x_2 + }
    }
    \IfEqCase{#4}{
        {0}{\overline{x_3}}
        {1}{x_3}
    }[\PackageError{minTerm}{Values must either be 0 or 1}]
    )
}

\begin{document}
    \maketitle
    \section{Digitale Schaltungen}
    \subsection{}
    \begin{enumerate}[label = (\alph*)]
        \item 
            \begin{table}[H]
                \centering
                \begin{tabular}{cccc|c|c}
                    \toprule
                    $2^3 = x_3$ & $2^2 = x_2$ & $2^1 = x_1$ & $2^0 = x_0$ & $x_{10}$ & $f(x)$ \\
                    \midrule
                    0 & 0 & 0 & 0 & 0 & 1\\
                    0 & 0 & 0 & 1 & 1 & 1\\
                    0 & 0 & 1 & 0 & 2 & 1\\
                    0 & 0 & 1 & 1 & 3 & 1\\
                    0 & 1 & 0 & 0 & 4 & 1\\
                    0 & 1 & 0 & 1 & 5 & 1\\
                    0 & 1 & 1 & 0 & 6 & 1\\
                    0 & 1 & 1 & 1 & 7 & 0\\
                    1 & 0 & 0 & 0 & 8 & 1\\
                    1 & 0 & 0 & 1 & 9 & 1\\
                    1 & 0 & 1 & 0 & 10 & 1\\
                    1 & 0 & 1 & 1 & 11 & 0\\
                    1 & 1 & 0 & 0 & 12 & 1\\
                    1 & 1 & 0 & 1 & 13 & 1\\
                    1 & 1 & 1 & 0 & 14 & 1\\
                    1 & 1 & 1 & 1 & 15 & 1\\
                    \bottomrule
                \end{tabular}
            \end{table}
        \item
            \begin{eqnarray*}
                f(x) &=& \minTerm{0}{0}{0}{0}\\
                    && + \minTerm{0}{0}{0}{1}\\ 
                    && + \minTerm{0}{0}{1}{0}\\ 
                    && + \minTerm{0}{0}{1}{1}\\ 
                    && + \minTerm{0}{1}{0}{0}\\ 
                    && + \minTerm{0}{1}{0}{1}\\ 
                    && + \minTerm{0}{1}{1}{0}\\ 
                    && + \minTerm{1}{0}{0}{0}\\ 
                    && + \minTerm{1}{0}{0}{1}\\ 
                    && + \minTerm{1}{0}{1}{0}\\ 
                    && + \minTerm{1}{1}{0}{0}\\ 
                    && + \minTerm{1}{1}{0}{1}\\ 
                    && + \minTerm{1}{1}{1}{0}\\ 
                    && + \minTerm{1}{1}{1}{1}\\ 
            \end{eqnarray*}
        \item
            \begin{equation*}
                f(x) = \maxTerm{1}{0}{0}{0} \cdot \maxTerm{0}{1}{0}{0}
            \end{equation*}
        \item %TODO
        \item
            Karnaugh-Veitch-Diagramm:
            \begin{figure}[H]
                \centering
                \begin{tabular}{cc|cccc|cc}
                    & &  & \multicolumn{2}{c}{$x_0$} & & \\
                    & & 0 & 1 & 1 & 0\\
                    \midrule
                    \multirow{4}{*}{$x_1$} & 0 & \cellcolor{purple!50}1 & \cellcolor{purple!50}1 & \cellcolor{blue!50}1 & \cellcolor{purple!50}1 & 0 &\multirow{4}{*}{$x_3$}\\
                     & 1 & \cellcolor{purple!50}1 & \cellcolor{purple!50}1 & 0 & \cellcolor{red!50}1 & 0\\
                     & 1 & \cellcolor{green!50}1 & 0 & \cellcolor{orange!50}1 & \cellcolor{orange!50}1 & 1\\
                     & 0 & \cellcolor{cyan}1 & \cellcolor{yellow!50}1 & \cellcolor{orange!50}1 & \cellcolor{orange!50}1 & 1\\
                    \midrule
                    & & 0 & 0 & 1 & 1\\
                    & &  & \multicolumn{2}{c}{$x_2$} & & \\
                \end{tabular}
            \end{figure}
            Daraus ergibt sich:
            \begin{eqnarray*}
                f_\text{Min}(x) &=& \textcolor{purple}{\overline{x_2} \cdot \overline{x_3}} \\ 
                    && + \textcolor{orange}{x_2 \cdot x_3} \\ 
                    && + \textcolor{blue}{\overline{x_1} \cdot x_2 \cdot \overline{x_3}} \\ 
                    && + \textcolor{red}{\overline{x_0} \cdot x_2 \cdot \overline{x_3}} \\ 
                    && + \textcolor{yellow}{\overline{x_1} \cdot \overline{x_2} \cdot x_3} \\ 
                    && + \textcolor{green}{\overline{x_0} \cdot \overline{x_2} \cdot x_3} \\ 
            \end{eqnarray*}
        \item %TODO
        \item %TODO
    \end{enumerate}

    \subsection{}
    \begin{enumerate}[label = (\alph*)]
        \item Im Falle des \glqq{}Durchschaltens\grqq{} des Logikgatters fließt n- bzw- p-mos Logikgattern ein Strom durch den Schaltenden Transitor. Bei C-Mos fließen nur minimale Ströme.
            Dadurch erhitzt sich die Schaltung nicht so sehr und es können mehr Gattter auf kleinerem Platz ohne thermische Probleme platziert werden.
    \end{enumerate}
\end{document}
