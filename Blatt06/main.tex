\documentclass[DIN, pagenumber=false, fontsize=11pt, parskip=half]{scrartcl}

\usepackage{amsmath}
\usepackage{amsfonts}
\usepackage{amssymb}
\usepackage{enumitem}
\usepackage[utf8]{inputenc} % this is needed for umlauts
\usepackage[ngerman]{babel} % this is needed for umlauts
\usepackage[T1]{fontenc} 
\usepackage{commath}
\usepackage{xcolor}
\usepackage{booktabs}
\usepackage{float}
\usepackage{tikz-timing}
\usepackage{tikz}
\usepackage{multirow}
\usepackage{colortbl}
\usepackage{xstring}
\usepackage[european]{circuitikz}
\usepackage{ifthen}

\usetikzlibrary{calc,shapes.multipart,chains,arrows}

\title{Grundlagen der Rechnerarchitektur}
\author{Tim Luchterhand, Paul Nykiel (Abgabegruppe 117)}

\newcommand{\minTerm}[4]{
    \IfEqCase{#1}{
        {0}{\overline{x_0}\ }
        {1}{x_0}
    }
    \IfEqCase{#2}{
        {0}{\overline{x_1}\ }
        {1}{x_1}
    }
    \IfEqCase{#3}{
        {0}{\overline{x_2}\ }
        {1}{x_2}
    }
    \IfEqCase{#4}{
        {0}{\overline{x_3}\ }
        {1}{x_3}
    }[\PackageError{minTerm}{Values must either be 0 or 1}]
}

\newcommand{\maxTerm}[4]{
    (
    \IfEqCase{#1}{
        {0}{\overline{x_0} + }
        {1}{x_0 + }
    }
    \IfEqCase{#2}{
        {0}{\overline{x_1} + }
        {1}{x_1 + }
    }
    \IfEqCase{#3}{
        {0}{\overline{x_2} + }
        {1}{x_2 + }
    }
    \IfEqCase{#4}{
        {0}{\overline{x_3}}
        {1}{x_3}
    }[\PackageError{minTerm}{Values must either be 0 or 1}]
    )
}


\begin{document}
    \maketitle
    \section{Digitale Schaltungen}
    \subsection{}
    \begin{enumerate}[label = (\alph*)]
        \item 
            \begin{table}[H]
                \centering
                \begin{tabular}{cccc|c|c}
                    \toprule
                    $2^3 = x_3$ & $2^2 = x_2$ & $2^1 = x_1$ & $2^0 = x_0$ & $x_{10}$ & $f(x)$ \\
                    \midrule
                    0 & 0 & 0 & 0 & 0 & 1\\
                    0 & 0 & 0 & 1 & 1 & 1\\
                    0 & 0 & 1 & 0 & 2 & 1\\
                    0 & 0 & 1 & 1 & 3 & 1\\
                    0 & 1 & 0 & 0 & 4 & 1\\
                    0 & 1 & 0 & 1 & 5 & 1\\
                    0 & 1 & 1 & 0 & 6 & 1\\
                    0 & 1 & 1 & 1 & 7 & 0\\
                    1 & 0 & 0 & 0 & 8 & 1\\
                    1 & 0 & 0 & 1 & 9 & 1\\
                    1 & 0 & 1 & 0 & 10 & 1\\
                    1 & 0 & 1 & 1 & 11 & 0\\
                    1 & 1 & 0 & 0 & 12 & 1\\
                    1 & 1 & 0 & 1 & 13 & 1\\
                    1 & 1 & 1 & 0 & 14 & 1\\
                    1 & 1 & 1 & 1 & 15 & 1\\
                    \bottomrule
                \end{tabular}
            \end{table}
        \item
            \begin{eqnarray*}
                f(x) &=& \minTerm{0}{0}{0}{0}\\
                    && + \minTerm{0}{0}{0}{1}\\ 
                    && + \minTerm{0}{0}{1}{0}\\ 
                    && + \minTerm{0}{0}{1}{1}\\ 
                    && + \minTerm{0}{1}{0}{0}\\ 
                    && + \minTerm{0}{1}{0}{1}\\ 
                    && + \minTerm{0}{1}{1}{0}\\ 
                    && + \minTerm{1}{0}{0}{0}\\ 
                    && + \minTerm{1}{0}{0}{1}\\ 
                    && + \minTerm{1}{0}{1}{0}\\ 
                    && + \minTerm{1}{1}{0}{0}\\ 
                    && + \minTerm{1}{1}{0}{1}\\ 
                    && + \minTerm{1}{1}{1}{0}\\ 
                    && + \minTerm{1}{1}{1}{1}\\ 
            \end{eqnarray*}
        \item
            \begin{equation*}
                f(x) = \maxTerm{1}{0}{0}{0} \cdot \maxTerm{0}{1}{0}{0}
            \end{equation*}
        \item Gatterschaltung: 
            Folgende Gatterschaltung zeigt nur den Aufbau für einen der Minterme (konkret für den ersten Minterm). Bei den anderen Mintermen
            werden logischerweise nicht alle Eingänge invertiert. Die komplette Schaltung erhält man, indem man alle diese Minterme mit einem großen
            ODER-Gatter zusammenschaltet. Da diese Schaltung gigantisch wird, ist sie hier nicht vollständig abgebildet.
            \begin{figure}[H]
                \centering
                \begin{circuitikz}
                    \node[european and port] at (6,0.5) (and00){};
                    \node[european and port] at (6,2.5) (and01){};
                    \node[european and port] at (8,1.5) (and02){};
                    \node[european not port] at (3,0) (not00){};
                    \node[european not port] at (3,1) (not01){};
                    \node[european not port] at (3,2) (not02){};
                    \node[european not port] at (3,3) (not03){};
                    \draw (and00.out) -| (and02.in 2);
                    \draw (and01.out) -| (and02.in 1);
                    \draw (not00.out) -| (and00.in 2);
                    \draw (not01.out) -| (and00.in 1);
                    \draw (not02.out) -| (and01.in 2);
                    \draw (not03.out) -| (and01.in 1);
                \end{circuitikz}
            \end{figure}
        \item
            Aus der Logiktabelle ergibt sich:
            Die Funktion ist nur bei $x_{10}=7 \lor x_{10}=11$ null. Hierbei gilt in beiden Fällen $x_0 = x_1 = 1$ und $x_2 = 1 \ \dot{\lor} \ x_3 = 1$.
            Daraus folgt für die Funktion:
            \begin{equation*}
                f_\text{Min} = \overline{(x_2 \cdot x_3) \cdot (x_0 \oplus x_1)}
            \end{equation*}
        \item Gatterschaltung:
            \begin{figure}[H]
                \centering
                \begin{circuitikz}
                    \node at (0, 0) (x0) {$x_0$};
                    \node at (0, 1) (x1) {$x_1$};
                    \node at (0, 2) (x2) {$x_2$};
                    \node at (0, 3) (x3) {$x_3$};
                    \node[european xor port] at (3,0.5) (xor){};
                    \node[european and port] at (3,2.5) (and){};
                    \node[european nand port] at (6,1.5) (nand){};
                    \draw (x0) -| (xor.in 2);
                    \draw (x1) -| (xor.in 1);
                    \draw (x2) -| (and.in 2);
                    \draw (x3) -| (and.in 1);
                    \draw (xor.out) -| (nand.in 2);
                    \draw (and.out) -| (nand.in 1);
                    \node at (8, 1.5) (f) {$f(x)$};
                    \draw (nand.out) -| (7,1.5);
                \end{circuitikz}
            \end{figure}
        \item C-Mos-Schaltung
            \begin{figure}[H]
                \centering
                \begin{circuitikz}
                    %AND
                    \node at (0, 2) (x2) {$x_2$};
                    \node at (0, 4) (x3) {$x_3$};
                    \node [pmos] at (2,6)(andp0){};
                    \node [nmos] at (2,4)(andn0){};
                    \node [nmos] at (2,2)(andn1){};
                    \node [pmos] at (4,8)(andp1){};
                    \node [rground] at (2,0)(gnd0){};
                    \draw (x2) -| (andn1.G);
                    \draw (x3) -| (andn0.G);
                    \draw (x3) |- (andp0.G);
                    \draw (x2) to[short, -*] (1,2) |- (andp1.G);
                    \draw (andp0.D) -- (andn0.D);
                    \draw (andn0.S) -- (andn1.D);
                    \draw (2,5) to[short, -*] (5,5);
                    \draw (andp1.D) to[short, -*] (4,5);
                    \draw (gnd0) -| (andn1.D);
                    \node [vcc] at (2,9) (vcc0){};
                    \draw (andp0.S) to[short, -*] (vcc0);
                    \draw (andp1.S) |- (vcc0);
                    \node [pmos] at (7,6)(andp2){};
                    \node [nmos] at (7,4)(andn2){};
                    \draw (5,5) |- (andp2.G);
                    \draw (5,5) |- (andn2.G);
                    \draw (andp2.D) |- (andn2.D);
                    \draw (andp2.S) |- (4,9);
                    \draw (4,9) to[short, -*] (4,9);
                    \draw (andn2.S) |- (gnd0);
                    \draw (gnd0) to[short, -*] (gnd0);
                    \draw (7,5) to[short,*-] (9,5);

                    %XOR
                    \node [pmos] at (2, 16)(xorp0){};
                    \node [pmos] at (6, 16)(xorp1){};
                    \node [nmos] at (2, 14)(xorn0){};
                    \node [nmos] at (6, 14)(xorn1){};
                    \node at (0,15)(x0){$x_0$};
                    \node at (4,15)(x1){$x_1$};
                    \node [vcc] at (2,18)(vcc1){};
                    \node [rground] at (2,12)(gnd1){};
                    \draw (vcc1) -| (xorp0.S);
                    \draw (xorp0.D) -| (xorn0.D);
                    \draw (xorn0.S) -| (gnd1);
                    \draw (x0) -| (xorp0.G);
                    \draw (x0) -| (xorn0.G);
                    \draw (x1) -| (xorp1.G);
                    \draw (x1) -| (xorn1.G);
                    \draw (x0) |- (xorp1.S);
                    \draw (2,15) to[short,*-] (3,15) |- (xorn1.S);
                    \draw (xorp1.D) |- (xorn1.D);
                    \draw (6,15) -| (8,15);

                    % NAND
                    \node [pmos] at (10,12)(nandp0){};
                    \node [pmos] at (12,12)(nandp1){};
                    \node [nmos] at (11,10)(nandn0){};
                    \node [nmos] at (11,8)(nandn1){};
                    \node [vcc] at (12,14)(vcc2){};
                    \node [rground] at (11,7)(gnd2){};
                    
                    \draw (nandp0.D) |- (nandn0.D);
                    \draw (nandp1.D) |- (nandn0.D);
                    \draw (nandn0.S) |- (nandn1.D);
                    \draw (nandn1.S) |- (gnd2);
                    \draw (nandp1.S) |- (vcc2);
                    \draw (nandp0.S) |- (12,13) to[short,*-] (vcc2);
                    \draw (8,15) |- (nandn0.G);
                    \draw (9,5) |- (nandn1.G);
                    \draw (9,10) to[short,*-] (9,10);
                    \draw (9,10) |- (nandp0.G);
                    \draw (8,15) -| (nandp1.G);
                    \draw (8,15) to[short,-*] (8,15);
                    \draw (11,9) to[short,*-] (13,9);
                    \node at (14,9){$f(x)$};
                \end{circuitikz}
            \end{figure}
    \end{enumerate}

    \subsection{}
    \begin{enumerate}[label = (\alph*)]
        \item Bei n- bzw. p-Mos Schaltungen fließt in dem Fall, dass der Ausgang der Schaltung auf
            \textit{GND} liegt, dauerhaft ein Strom durch den Pull-Up Widerstand.
            Dieser Strom sorgt für einen erhöhten Energieverbrauch und damit zu 
            thermischen Problemen, wenn viele Logikelemente auf kleinem Raum realisiert werden.

            Zudem muss eine Kompromiss bei der Dimensionierung des Pull-Up Widerstandes
            eingegangen werden: Bei zu kleinem Widerstand fließt ein sehr großer Strom,
            bei zu großem Widerstand kann das Gatter nicht mehr funktionieren wenn
            der Ausgang gegen \textit{GND} gezogen wird.

            Des weiteren ist es schwierig, Widerstände auf die Größenordnung von MOS-Transitoren
            zu verkleineren. Außerdem führt die vergleichsweise große physikalische Ausdehnung
            von Widerständen bei hohen Taktraten zu Problemen durch die parasitären
            Kapazitäten und Induktivitäten eines Widerstandes.
        \item
            Die ausprobieren der verschiedenen Eingangswerte ergibt sich:
            \begin{table}[H]
                \centering
                \begin{tabular}{ccc|c}
                    \toprule
                    $x_1$ & $x_2$ & $x_3$ & $f(x_1, x_2, x_3)$ \\
                    \midrule
                    0 & 0 & 0 & 1 \\
                    1 & 0 & 0 & 1 \\
                    0 & 1 & 0 & 1 \\
                    1 & 1 & 0 & 0 \\
                    0 & 0 & 1 & 1 \\
                    1 & 0 & 1 & 1 \\
                    0 & 1 & 1 & 0 \\
                    1 & 1 & 1 & 0 \\
                    \bottomrule
                \end{tabular}
            \end{table}
            Daraus folgt:
            \begin{eqnarray*}
                f(x_1, x_2, x_3) &=& \overline{x_1 x_2 \overline{x_3} + \overline{x_1} x_2 x_3 + x_1 x_2 x_3} \\
                &=& \overline{x_1 x_2 \cdot (\overline{x_3} + x_3) + \overline{x_1} x_2 x_3} \\
                &=& \overline{x_1 \cdot x_2 + \overline{x_1} \cdot x_2 \cdot x_3}
            \end{eqnarray*}
        \item siehe b)
        \item 
            P-Mos Schaltung:
            \begin{figure}[H]
                \centering
                \begin{circuitikz}
                    \draw (2,2) to[R] (2,3);
                    \node [pmos] at (2, 6)(pmos0){};
                    \node [pmos] at (2, 8)(pmos1){};
                    \node [pmos] at (4, 10)(pmos2){};
                    \node [pmos] at (4, 12)(pmos3){};
                    \node [pmos] at (6, 10)(pmos4){};
                    \node [pmos] at (6, 12)(pmos5){};
                    \node [pmos] at (8, 12)(pmos6){};
                    \node [pmos] at (8, 14)(pmos7){};
                    \node [pmos] at (10, 5)(pmos8){};
                    \node [pmos] at (10, 7)(pmos9){};
                    \node [pmos] at (10, 9)(pmos10){};
                    \draw (2,4) to (11,4);
                    \draw (2,15) to (10,15);
                    \node [vcc] at (2,16)(vcc){};
                    \node [rground] at (2,0)(gnd){};
                    \draw (vcc) -- (2,15);
                    \draw (gnd) -- (2,2);
                    \node at (12,4) {$f(x)$};
                    \node at (0,5) (x1){$x_1$};
                    \node at (0,9) (x2){$x_2$};
                    \node at (0,14) (x3){$x_3$};

                    \draw (2,3) to[short, -*] (2,4);
                    \draw (2,4) |- (pmos0.D);
                    \draw (pmos0.S) |- (pmos1.D);
                    \draw (pmos1.S) to[short, -*]  (2,15);

                    \draw (4,4) to[short, *-]  (pmos2.D);
                    \draw (pmos2.S) |- (pmos3.D);
                    \draw (pmos3.S) to[short, -*]  (4,15);

                    \draw (6,4) to[short, *-]  (pmos4.D);
                    \draw (pmos4.S) |- (pmos5.D);
                    \draw (pmos5.S) to[short, -*]  (6,15);

                    \draw (8,4) to[short, *-]  (pmos6.D);
                    \draw (pmos6.S) |- (pmos7.D);
                    \draw (pmos7.S) to[short, -*]  (8,15);

                    \draw (10,4) to[short, *-]  (pmos8.D);
                    \draw (pmos8.S) |- (pmos9.D);
                    \draw (pmos9.S) to[short, -*]  (10,15);

                    \draw (x1) -| (pmos0.G);
                    \draw (x2) -| (pmos1.G);
                    \draw (x2) to[short, -*] (1,9) |- (pmos2.G);
                    \draw (x3) |- (pmos3.G);
                    \draw (x2) |- (5, 9) -| (pmos4.G);
                    \draw (6,13)  -| (pmos5.G);
                    \draw (x3) -| (pmos7.G);
                    \draw (6,13) to[short, *-] (7,13) -| (pmos6.G);
                    \draw (2,5) to[short, *-] (pmos8.G);
                    \draw (4,7) to[short, *-] (pmos9.G);
                    \draw (8,9) to[short, *-] (pmos10.G);
                \end{circuitikz}
            \end{figure}
    \end{enumerate}
\end{document}
